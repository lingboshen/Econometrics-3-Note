\documentclass{article}
\usepackage{comment}
\usepackage{amssymb}
\usepackage{mathrsfs}
\usepackage{amsmath,amsthm,amssymb,amsfonts}
\usepackage{amsfonts}
\usepackage{bm}
\usepackage{commath}
\usepackage{pgf,tikz}
\usetikzlibrary{arrows}
\pagestyle{empty}
\usepackage{indentfirst}
\usepackage[margin=3.5cm]{geometry}
\linespread{1.45}
\usepackage[colorlinks,linkcolor=red,anchorcolor=blue,citecolor=green]{hyperref}
\newtheorem{remark}{Remark}
\newtheorem{theorem}{Theorem}
\newtheorem{definition}{Definition}
\DeclareMathOperator*{\plim}{plim}

\newcommand{\tmop}[1]{\ensuremath{\operatorname{#1}}}
\newenvironment{enumeratenumeric}{\begin{enumerate}[1.] }{\end{enumerate}}
\newenvironment{enumerateroman}{\begin{enumerate}[i.] }{\end{enumerate}}
%\newenvironment{proof}{\noindent\textbf{Proof\ }}{\hspace*{\fill}$\Box$\medskip}
%%%%%%%%%% End TeXmacs macros

\newcommand{\x}{\bm{x}}


\newcommand*{\MyPath}{./Chapter}


\begin{document}




\title{Time Series Note}
\author{Lingbo Shen}
\maketitle

\tableofcontents
\newpage

%model
\input{\MyPath/ch1}

%lag operator
\input{\MyPath/ch2}

%arma
\input{\MyPath/ch3}

%asymptotic for stationary
\section[Asymptotics for Dynamic Models]{Asymptotics for Dynamic Models}
\subsection{The Simple Autoregressive Model}


\subsection{Martingale Difference Processes}
The adapted sequence $\{\bm{s}_{t},\mathcal{F}_{t}\}$ is called a martingale if for every $t$ the following conditions hold:
\begin{enumerate}
\item $E|\bm{s}_{t}|<\infty$
\item $E(\bm{s}_{t}\mid\mathcal{F}_{t-1})=\bm{s}_{t-1}$ a.s.
\end{enumerate}

The adapted sequence $\{\bm{x}_{t},\mathcal{F}_{t}\}$ is called a martingale difference (m.d.) if for every $t$ the following conditions hold:
\begin{enumerate}
\item $E|\bm{x}_{t}|<\infty$
\item $E(\bm{s}_{t}\mid\mathcal{F}_{t-1})=0$ a.s.
\end{enumerate}

Here is a fundamental property of m.d. processes.
\begin{theorem}
Let $\{x_{t}\}$ be an m.d. sequence and let $g_{t-1}=g(x_{t-1},x_{t-2},\cdots,)$ be any measurable, integrable function of the lagged values of the sequence. Then $x_{t}g_{t-1}$ is also an m.d., and $x_{t}$ and $g_{t-1}$ are uncorrelated.
\end{theorem}
\begin{proof}
By law of iterated expectations, we have
\begin{eqnarray*}
E(x_{t}g_{t-1})&=&E(E(x_{t}g_{t-1}\mid\mathcal{F}_{t-1}))\\
			&=&E(g_{t-1}E(x_{t}\mid\mathcal{F}_{t-1}))\\
			&=&E(g_{t-1}\cdot 0)\\
			&=&0
\end{eqnarray*}
It means that $x_{t}g_{t-1}$ is also an m.d. For uncorrelation, we show that
\begin{eqnarray*}
Cov(x_{t},g_{t-1})&=&E(x_{t}g_{t-1})-E(x_{t})E(g_{t-1})\\
			   &=&E(x_{t}g_{t-1})-E(E(x_{t}\mid\mathcal{F}_{t-1})E(g_{t-1})\\
			  &=&0-0\\
			&=&0
\end{eqnarray*}
\end{proof}

In particular, putting $g_{t-1}=x_{t-j}$ for any $j>0$, the theorem implies 
\begin{eqnarray*}
Cov(x_{t},x_{t-j})&=&0
\end{eqnarray*}
\textbf{The m.d. property implies uncorrelatedness of the sequence, although it is a weaker property than independence.} 

Given an arbitrary sequence $\{y_{t}\}$ satisfying $E|y_{t}|<\infty$ and $\sigma\left(y_{t},y_{t-1},\cdots\right)\subset \mathcal{F}_{t}$, an m.d. sequence can always be generated as the centred sequence $\{x_{t}\}$ where 
\begin{eqnarray*}
x_{t}&=&y_{t}-E(y_{t}\mid\mathcal{F}_{t-1})
\end{eqnarray*}

We consider the law of large number and central limit theorem for m.d. processes.
\begin{theorem}
Let $\{x_{t}\}$ be an m.d. sequence. Then $\plim \bar{x}_{n}=0$ if either of the following condition hold.
\begin{enumerate}
\item The sequence is strictly stationary and $E|x_{t}|<\infty$
\item $E|x_{t}|^{1+\delta}<\infty$, $\forall t$
\end{enumerate}
\end{theorem}

The following is central limit theorem for m.d.
\begin{theorem}
Let $\{x_{t},\mathcal{F}_{t}\}$ be a m.d. sequence with $E(x_{t}^{2})=\sigma_{t}^{2}$ and let $\bar{\sigma}_{n}^{2}=n^{-1}\sum_{t=1}^{n}\sigma_{t}^{2}$. If
\begin{enumerate}
\item  $\plim n^{-1}\sum_{t=1}^{n}\left(x_{t}^{2}-\sigma_{t}^{2}\right)=0$
\item either 
	\begin{enumerate}
	\item the sequence is strictly stationary
	\item 
	\end{enumerate}
\end{enumerate}
\end{theorem}




%unit root
\input{\MyPath/ch6}


%var
\input{\MyPath/ch8}

\begin{comment}
\end{comment}



\section[Trends and Unit Roots]{Trends and Unit Roots}
\subsection{The Random Walk Model}
Consider the following special case AR(1) process:
\begin{eqnarray*}
y_{t}&=&y_{t-1}+u_{t}\\
\Delta y_{t}&=&u_{t}
\end{eqnarray*}

If $y_{0}$ is a given initial condition, its solution is
\begin{eqnarray*}
y_{t}&=&y_{0}+\sum_{i=1}^{t}u_{i}
\end{eqnarray*}

Take expectation
\begin{eqnarray*}
Ey_{t}&=& E\left(y_{0}+\sum_{i=1}^{t}u_{i}\right)=y_{0}
\end{eqnarray*}
Taking expectation and variance
\begin{eqnarray*}
E_{t}y_{t+1}&=&E_{t}\left(y_{t}+\varepsilon_{t+1}\right)=y_{t}\\
E_{t}y_{t+s}&=&E_{t}\left(y_{t}+\sum_{i=1}^{s}\varepsilon_{t+i}\right)=y_{t}\\
Var(y_{t})&=&Var\left(\sum_{i=1}^{t}u_{i}\right)=t\sigma^{2}\\
Var(y_{t-s})&=&Var\left(\sum_{i=1}^{t-s}u_{i}\right)=(t-s)\sigma^{2}\\
E\left[(y_{t}-y_{0})(y_{t-s}-y_{0})\right]&=&E\left[(\sum_{i=1}^{t}u_{i} )(\sum_{i=1}^{t+s}u_{i})\right]\\
								&=&E\left[(\sum_{i=1}^{t-s}u_{i}^{2} )\right]\\
								&=&(t-s)\sigma^{2}
\end{eqnarray*}

The correlation coefficient $\rho_{s}$ is
\begin{eqnarray*}
\rho_{s}&=&\frac{(t-s)\sigma^{2}}{\sqrt{(t-s)\sigma^{2}}\times \sqrt{t\sigma^{2}}}\\
&=&\sqrt{\frac{t-s}{t}}
\end{eqnarray*}

When $t$ is big relative to s, the $\rho_{s}$  are close to unity and decay very slowly. 

\paragraph{The Random Walk plus Drift Model}
Adding a constant term $a_{0}$:
\begin{eqnarray*}
y_{t}&=&y_{t-1}+a_{0}+u_{t}
\end{eqnarray*}

Giving the initial condition $y_{0}$, its solution is
\begin{eqnarray*}
y_{t}&=&y_{0}+a_{0}t+\sum_{i=1}^{t}u_{i}
\end{eqnarray*}

The behavior of $y_{t}$ is governed by two nonstationary components: a linear deterministic trend and the stochastic trend. 

\subsection{Function Spaces}
\begin{itemize}
\item $x$: an element of $C$, that is, any continuous curve traversing the unit interval, be denoted.
\item Coordinates of $x$: $x(r)\in\mathbb{R}$ is the unique values of $x$ at points $r\in[0,1]$ are called the coordinates of $x$.
\end{itemize}

For two members of $C$, $x\in C$ and $y\in C$, we need to say how close together they are. Technically, $C$ must be assigned a metric. For example, we can define \textbf{Euclidean metric} for any pair of real numbers $x$ and $y$ as $d_{E}(x,y)=|x-y|$. The pair $(\mathbb{R},d_{E})$ is known as the \textbf{Euclidean space}. 
 
We also can define a metric called \textbf{uniform metric} as 
\begin{eqnarray*}
d_{U}(x,y)&=&\sup_{0\leq r\leq 1}|x(r)-y(r)|
\end{eqnarray*}
This is just the \textit{largest vertical separation} between the pair of functions over the interval. $(C,d_{U})$ is a metric space. 

\subsection{Brownian Motion}
A Brownian motion $B$ is a real random function on the unit interval, with the following properties:
\begin{enumerate}
\item $B\in C$ with probability 1.
\item $B(0)=0$ with probability 1.
\item for any set of subintervals defined by arbitrary $0\leq r_{1}<r_{2}<\dots<r_{k}\leq 1$, the increments $B(r_{1}$, $B(r_{2})-B(r_{1}$, $\cdots$, $B(r_{k})-B(r_{k-1})$ are independent.
\item $B(t)-B(s)\sim N(0,t-s)$ for $0\leq s<t\leq 1$.
\end{enumerate}

\subsection{The Functional Central Limit Theorem}
We construct a variable $X_{T}(r)$ from the sample mean of the first $r$th fraction of the observations, $r\in[0,1]$, defined by
\begin{eqnarray*}
X_{T}(r)&\equiv&\frac{1}{T}\sum_{t=1}^{[Tr]}u_{t}
\end{eqnarray*}
For any given realization, $X_{T}(r)$ is a step function in $r$, with
\begin{eqnarray*}
X_{T}(r)&=&\begin{cases}
0						&	0\leq r<1/T\\
\frac{u_{1}}{T}				&	1/T\leq r<2/T\\
\frac{u_{1}+u_{2}}{T}		&	2/T\leq r<3/T\\
\vdots					&				\\
\frac{u_{1}+u_{2}+\cdots+u_{T-1}}{T}		&	(T-1)/T\leq r<1\\
\frac{u_{1}+u_{2}+\cdots+u_{T-1}+u_{T}}{T}		&	r=1\\
\end{cases}
\end{eqnarray*}
Then 
\begin{eqnarray*}
\sqrt{T}X_{T}(r)&=&\sqrt{T}\frac{1}{T}\sum_{t=1}^{[Tr]}u_{t}\\
			&=&\frac{1}{\sqrt{T}}\sum_{t=1}^{[Tr]}u_{t}\\
			&=&\frac{\sqrt{[Tr]}}{\sqrt{T}}\frac{1}{\sqrt{[Tr]}}\sum_{t=1}^{[Tr]}u_{t}\\
			&\approx_{d}&\sqrt{r} N(0,\sigma^{2})\\
			&\sim&N(0,r\sigma^{2})
\end{eqnarray*}
The last second comes from the facts that $\frac{\sqrt{[Tr]}}{\sqrt{T}}\to \sqrt{r}$ and $\frac{1}{\sqrt{[Tr]}}\sum_{t=1}^{[Tr]}u_{t}\approx_{d}N(0,\sigma^{2})$.

If we consider a sample mean based on observations $[Tr_{1}]$ through $[Tr_{2}]$ for $r_{2}>r_{1}$, we would conclude that 
\begin{eqnarray*}
\sqrt{T}\left[X_{T}(r_{2})-X_{T}(r_{2})\right]/\sigma&\approx&N(0,r_{2}-r_{1})
\end{eqnarray*}

It should not be surprising that the sequence of stochastic functions $\left\{\sqrt{T}X_{T}(\cdot)/\sigma\right\}_{T=1}^{\infty}$ has an asymptotic probability law that is described by standard Brownian motion
\begin{eqnarray*}
\sqrt{T}X_{T}(\cdot)/\sigma&\to&W(\cdot)
\end{eqnarray*}
This result is called \textit{functional central limit theorem}.
\begin{remark}
We should notice the difference between $X_{T}(\cdot)$ and $X_{T}(r)$. $X_{T}(\cdot)$ denotes a random function while $X_{T}(r)$ denotes the value that function assumes at date $r$; thus, $X_{T}(\cdot)$ is a function, while $X_{T}(r)$ is a random variable.
\end{remark}

\paragraph{Convergence in law for random functions}
Let $S(\cdot)$ be a continuous-time stochastic process with $S(r)$ representing its value at time date $r$ for $r\in [0,1]$. Suppose that for any given realization $S(\cdot)$ is a continuous function of $r$ with probability 1. 
\begin{definition}
For $\left\{S_{T}(\cdot)\right\}_{T=1}^{\infty}$ a sequence of such continuous functions, we say that $S_{T}(\cdot)\xrightarrow{L}S(\cdot)$ if all of the following hold:
\begin{enumerate}
\item For any finite collection of $k$ particular dates $0\leq r_{1}<r_{2}<\cdots<r_{k}\leq 1$, the sequence of $k-$dimensional random vectors $\{\bm{y}_{T}\}_{T=1}^{\infty}$ converges in distribution to the vector $\bm{y}$, where 
	\begin{eqnarray*}
	\bm{y}_{T}\equiv\left[\begin{array}{c}
	S_{T}(r_{1})\\
	S_{T}(r_{2})\\
	\vdots\\
	S_{T}(r_{k})\\
	\end{array}\right]
	&&
	\bm{y}\equiv\left[\begin{array}{c}
	S(r_{1})\\
	S(r_{2})\\
	\vdots\\
	S(r_{k})\\
	\end{array}\right]
	\end{eqnarray*}
\item For each $\varepsilon>0$, the probability that $S_{T}(r_{1})$ differs from $S_{T}(r_{2})$ for any dates $r_{1}$ and  $r_{2}$ within $\delta$ of each other goes to zero uniformly in $T$ as $\delta\to 0$.
\item $P\{|S_{T}(0)|>\lambda\}\to 0$ uniformly in $T$ as $\lambda\to\infty$.
\end{enumerate}
\end{definition}
\begin{remark}
This definition applies to sequences of continuous functions, though the function $X_{T}(r)$ is a discontinuous step function. Fortunately, the discontinuities occur at a countable set of points.
\end{remark}
\paragraph{Convergence in probability to sequences of random functions} Let $\left\{S_{T}(\cdot)\right\}_{T=1}^{\infty}$ and $\left\{V_{T}(\cdot)\right\}_{T=1}^{\infty}$ denote sequences of random continuous functions with $S_{T}:r\in [0,1]\to\mathbb{R}^{1}$ and $V_{T}:r\in [0,1]\to\mathbb{R}^{1}$. Let the scalar $Y_{T}$ represent the largest amount by which $S_{T}(r)$ differs from $V_{T}(r)$ for any $r$ defined by
\begin{eqnarray*}
Y_{T}&\equiv&\sup_{0\leq r\leq 1}|S_{T}(r)-V_{T}(r)|
\end{eqnarray*}
\begin{definition}
If the sequence of scalars $\left\{Y_{T}(\cdot)\right\}_{T=1}^{\infty}$ converges in probability to zero, then we say that the sequence of functions $S_{T}(\cdot)$ converges in probability to $V_{T}(\cdot)$. That is, the expression
\begin{eqnarray*}
S_{T}(\cdot)&\to_{p}&V_{T}(\cdot)
\end{eqnarray*}
is interpreted to mean that
\begin{eqnarray*}
\sup_{r\in[0,1]}|S_{T}(r)-V_{T}(r)|&\to_{p}&0
\end{eqnarray*}
\end{definition}


\paragraph{Continuous mapping theorem}
The result is similar to Slutsky's theorem. 
\begin{theorem}
The continuous mapping theorem states that if $S_{T}(\cdot)\xrightarrow{L}S(\cdot)$ and $g(\cdot)$ is a continuous \textbf{functional}, then $g\left(S_{T}(\cdot)\right)\xrightarrow{L}g\left(S(\cdot)\right)$.
\end{theorem}
For example, consider $S_{T}(r)\equiv [\sqrt{T}X_{T}(r)]^{2}$ and $\sqrt{T}X_{T}(r)\xrightarrow{L}\sigma W(\cdot)$. It follows that
\begin{eqnarray*}
S_{T}(\cdot)&\xrightarrow{L}&\sigma^{2}[W(\cdot)]^{2}
\end{eqnarray*}

\subsection{Unit Root Processes}
A random walk $y_{t}=y_{t-1}+u_{t}$ where $\{u_{t}\}$ is an i.i.d. sequence with zero mean and variance $\sigma^{2}$. If $y_{0}=0$, then 
\begin{eqnarray*}
y_{t}=u_{1}+u_{2}+\cdots+u_{t}
\end{eqnarray*}
The stochastic function $X_{T}(r)$ defined as
\begin{eqnarray*}
X_{T}(r)&=&\begin{cases}
0						&	0\leq r<1/T\\
\frac{y_{1}}{T}				&	1/T\leq r<2/T\\
\frac{y_{2}}{T}				&	2/T\leq r<3/T\\
\vdots					&				\\
\frac{y_{T-1}}{T}			&	(T-1)/T\leq r<1\\
\frac{y_{T}}{T}				&	r=1\\
\end{cases}
\end{eqnarray*}
The figure below plots $X_{T}(r)$ as a function of $r$. The area under this step function is the sum of $T$ rectangles. The integral of $X_{T}(r)$ is thus equivalent to 
\begin{eqnarray*}
\int_{0}^{1}X_{T}(r)dr&=&\frac{y_{1}}{T^{2}}+\frac{y_{2}}{T^{2}}+\cdots+\frac{y_{T-1}}{T^{2}}
\end{eqnarray*}
Then we have
\begin{center}
\definecolor{xdxdff}{rgb}{0.490196078431,0.490196078431,1.}
\definecolor{uuuuuu}{rgb}{0.266666666667,0.266666666667,0.266666666667}
\begin{tikzpicture}[line cap=round,line join=round,>=triangle 45,x=1.0cm,y=1.0cm]
\clip(-2.66,-1.02) rectangle (8.94,6);
\draw [->] (0.,0.) -- (0.,5.);
\node at (0,5.3) {$X_{T}(r)$};
\node at (8,-0.3) {$r$};
\draw [->] (0.,0.) -- (8.,0.);

\begin{scriptsize}
\draw [fill=uuuuuu] (0.,0.) ;
\draw [fill=xdxdff] (0.,6.) ;
\draw [fill=xdxdff] (8.,0.) ;
\draw (1,0) rectangle (2,3);
\node at (1,-0.3) {$\frac{1}{T}$};
\node at (1.5,1.3) {$\frac{y_{1}}{T}$};
\draw (2,0) rectangle (3,2.3);
\node at (2,-0.3) {$\frac{2}{T}$};
\node at (2.5,1.3) {$\frac{y_{2}}{T}$};
\draw (3,0) rectangle (4,3.2);
\node at (3,-0.3) {$\frac{3}{T}$};
\node at (3.5,1.3) {$\frac{y_{3}}{T}$};
\draw (4,0) rectangle (5,2.7);
\node at (4,-0.3) {$\frac{4}{T}$};
\draw (5,0) rectangle (6,3.5);
\node at (5,-0.3) {$\cdots$};
\end{scriptsize}
\end{tikzpicture}
\end{center}

\begin{eqnarray*}
\int_{0}^{1}\sqrt{T}X_{T}(r)dr&=&\sqrt{T}\left[\frac{y_{1}}{T^{2}}+\frac{y_{2}}{T^{2}}+\cdots+\frac{y_{T-1}}{T^{2}}\right]\\
						&=&T^{-3/2}\sum_{t=1}^{T}y_{t-1}
\end{eqnarray*}
By functional central limit theorem and continuous mapping theorem that as $T\to\infty$
\begin{eqnarray*}
\int_{0}^{1}\sqrt{T}X_{T}(r)dr&\xrightarrow{L}&\sigma\int_{0}^{1}W(r)dr
\end{eqnarray*}
which implies that
\begin{eqnarray}
\label{eq6.1}
T^{-3/2}\sum_{t=1}^{T}y_{t-1}&\xrightarrow{L}&\sigma\int_{0}^{1}W(r)dr
\end{eqnarray}

We can write
\begin{eqnarray}
\label{eq6.2}
\nonumber
T^{-3/2}\sum_{t=1}^{T}y_{t-1}&=&T^{-3/2}\left[u_{1}+(u_{1}+u_{2})+\cdots+(u_{1}+u_{2}+\cdots+u_{T-1})\right]\\\nonumber
	&=&T^{-3/2}\left[(T-1)u_{1}+(T-2)u_{2}+\cdots u_{T-1}\right]\\\nonumber
	&=&T^{-3/2}\sum_{t=1}^{T}(T-t)u_{t}\\
	&=&T^{-1/2}\sum_{t=1}^{T}u_{t}-T^{-3/2}\sum_{t=1}^{T}tu_{t}
\end{eqnarray}

From (\ref{eq6.2}), we can derive
\begin{eqnarray}
\label{eq6.3}
\nonumber
T^{-3/2}\sum_{t=1}^{T}tu_{t}&=&T^{-1/2}\sum_{t=1}^{T}u_{t}-T^{-3/2}\sum_{t=1}^{T}y_{t-1}\\
	&\to_{d}&\sigma W(1)-\sigma\int_{0}^{1}W(r)dr
\end{eqnarray}

\begin{theorem}
In summary, for random walk without drift $y_{t}=y_{t-1}+u_{t}$ where $y_{0}=0$ and $\{u_{t}\}$ is an i.i.d. sequence with mean zero and variance $\sigma^{2}$. Then we have

\begin{eqnarray}
\label{eq6.4}
T^{-1/2}\sum_{t=1}^{T}u_{t}&\xrightarrow{L}&\sigma W(1)\\
\label{eq6.5}
T^{-1}\sum_{t=1}^{T}y_{t-1}u_{t}&\xrightarrow{L}&\frac{1}{2}\sigma^{2} [W(1)^{2}-1]\\
\label{eq6.6}
T^{-3/2}\sum_{t=1}^{T}tu_{t}&\xrightarrow{L}&\sigma W(1)-\sigma\int_{0}^{1}W(r)dr\\
\label{eq6.7}
T^{-3/2}\sum_{t=1}^{T}y_{t-1}&\xrightarrow{L}&\sigma\int_{0}^{1}W(r)dr\\
\label{eq6.8}
T^{-2}\sum_{t=1}^{T}y_{t-1}^{2}&\xrightarrow{L}&\sigma^{2}\int_{0}^{1}W(r)^{2}dr\\
\label{eq6.9}
T^{-5/2}\sum_{t=1}^{T}ty_{t-1}&\xrightarrow{L}&\sigma\int_{0}^{1}rW(r)dr\\
\label{eq6.10}
T^{-3}\sum_{t=1}^{T}ty_{t-1}^{2}&\xrightarrow{L}&\sigma^{2}\int_{0}^{1}rW(r)^{2}dr\\
\label{eq6.11}
T^{-(v+1)}\sum_{t=1}^{T}t^{v}&\to&\frac{1}{v+1},\ \ \ v\in \mathbb{N}
\end{eqnarray}
\end{theorem}

\begin{proof}
(\ref{eq6.4}) is trivial. (\ref{eq6.6}) is shown in (\ref{eq6.3}). (\ref{eq6.7}) is shown in (\ref{eq6.1}). Now we prove the remaining properties. 

We define $S_{T}(r)$ as 
\begin{eqnarray*}
S_{T}(r)&=&\left(\sqrt{T}X_{T}(r)\right)^{2}=T[X_{T}(r)]^{2}
\end{eqnarray*}
Then it can be written as a step function 
\begin{eqnarray*}
S_{T}(r)&=&\begin{cases}
0							&	0\leq r<1/T\\
\frac{y_{1}^{2}}{T}				&	1/T\leq r<2/T\\
\frac{y_{2}^{2}}{T}				&	2/T\leq r<3/T\\
\vdots						&				\\
\frac{y_{T-1}^{2}}{T}			&	(T-1)/T\leq r<1\\
\frac{y_{T}^{2}}{T}				&	r=1\\
\end{cases}
\end{eqnarray*}
Then we have
\begin{eqnarray}
\label{eq4.12}
\int_{0}^{1}S_{T}(r)dr&=&\frac{y_{1}^{2}}{T^{2}}+\frac{y_{2}^{2}}{T^{2}}+\cdots+\frac{y_{T-1}^{2}}{T^{2}}=T^{-2}\sum_{t=1}^{T}y_{t-1}^{2}\xrightarrow{L}\sigma^{2}\int_{0}^{1}W(r)^{2}dr
\end{eqnarray}
Now we show (\ref{eq6.8}).

For (\ref{eq6.9}), we prove as
\begin{eqnarray*}
T^{-5/2}\sum_{t=1}^{T}ty_{t-1}&=&T^{-3/2}\sum_{t=1}^{T}\frac{t}{T}y_{t-1}\xrightarrow{L}\sigma\int_{0}^{1}rW(r)dr
\end{eqnarray*}
for $r=t/T$.

For (\ref{eq6.10})
\begin{eqnarray*}
T^{-3}\sum_{t=1}^{T}ty_{t-1}^{2}&=&T^{-2}\sum_{t=1}^{T}(t/T)y_{t-1}^{2}\xrightarrow{L}\sigma^{2}\int_{0}^{1}rW(r)^{2}dr
\end{eqnarray*}
For (\ref{eq6.5})
\begin{eqnarray*}
T^{-1}\sum_{t=1}^{T}y_{t-1}u_{t}&=&\frac{1}{2}\frac{1}{T}y_{T}^{2}-\frac{1}{2}\frac{1}{T}\sum_{t=1}^{T}u_{t}^{2}\\
&=&\frac{1}{2}S_{T}(1)-\frac{1}{2}\frac{1}{T}\sum_{t=1}^{T}u_{t}^{2}\\
&\xrightarrow{L}&\frac{1}{2}\sigma^{2}[W(1)]^{2}-\frac{1}{2}\sigma^{2}\\
&\xrightarrow{L}&\frac{1}{2}\sigma^{2} [W(1)^{2}-1]
\end{eqnarray*}
\end{proof}

\subsubsection{No Constant No Time Trend}
Consider OLS estimation of $\rho$ based on an AR(1) regression
\begin{eqnarray}
\label{eq4.13}
y_{t}&=&\rho y_{t-1}+u_{t}
\end{eqnarray}
where $u_{t}$ is i.i.d. with mean zero and variance $\sigma^{2}$. OLS estimate is
\begin{eqnarray*}
\hat{\rho}_{T}&=&\frac{\sum_{t=1}^{T}y_{t-1}y_{t}}{\sum_{t=1}^{T}y_{t-1}^{2}}
\end{eqnarray*}
when true value of $\rho$ is 1. Then we have
\begin{eqnarray*}
T\left(\hat{\rho}_{T}-1\right)&=&\frac{T^{-1}\sum_{t=1}^{T}y_{t-1}u_{t}}{T^{-2}\sum_{t=1}^{T}y_{t-1}^{2}}
\end{eqnarray*}
For random walk process $y_{t}=y_{0}+u_{1}+u_{2}+\cdots+u_{t}$, apart from the initial term $y_{0}$, by (\ref{eq6.5})
$$T^{-1}\sum_{t=1}^{T}y_{t-1}u_{t}\xrightarrow{L}\frac{1}{2}\sigma^{2} [W(1)^{2}-1]$$
and by (\ref{eq6.8})
$$T^{-2}\sum_{t=1}^{T}y_{t-1}^{2}\xrightarrow{L}\sigma^{2}\int_{0}^{1}W(r)^{2}dr$$
Under the null hypothesis that $\rho=1$, the OLS estimate is characterized by
\begin{eqnarray*}
T\left(\hat{\rho}_{T}-1\right)&=&\frac{T^{-1}\sum_{t=1}^{T}y_{t-1}u_{t}}{T^{-2}\sum_{t=1}^{T}y_{t-1}^{2}}\xrightarrow{L}\frac{\frac{1}{2}[W(1)^{2}-1]}{\int_{0}^{1}W(r)^{2}dr}
\end{eqnarray*}
$[W(1)]^{2}$ is a $\chi^{2}(1)$ variable. 


\subsubsection{Constant No Time Trend}
The DGP is still $y_{t}=y_{t-1}+u_{t}$. Suppose now that a constant term is included in the AR(1) model that is to be estimated by OLS
\begin{eqnarray*}
y_{t}&=&\alpha+\rho y_{t-1}+u_{t}
\end{eqnarray*}


\subsubsection{No Constant No Time Trend}

\subsection{Dickey-Fuller Tests}

\subsubsection{Serial Correlation}
When $\varepsilon_{t}$ is serially correlated, two approaches are developed. 
Subtracting $y_{t-1}$ from each side of the equation $y_{t}=a_{1}y_{t-1}+u_{t}$, we get $\Delta y_{t}=\gamma y_{t-1}+u_{t}$, where $\gamma=a_{1}-1$. Testing the hypothesis $a_{1}=1$ is equivalent to testing $\gamma=0$.

Dickey and Fuller consider three different regression equations 
\begin{eqnarray*}
&&\mbox{random walk model}\\
\Delta y_{t}&=&\gamma y_{t-1}+u_{t} \\
&&  \mbox{add a drift}\\
\Delta y_{t}&=&a_{0}+\gamma y_{t-1}+u_{t}\\
&&\mbox{add a drift and linear time trend}\\
\Delta y_{t}&=&a_{0}+\gamma y_{t-1}+a_{2}t+u_{t} 
\end{eqnarray*}

Run the OLS and get the estimated value of $\gamma$ and associated standard error of these three models. However, the critical values of the t-statistics do depend on whether a drift and/or time trend is included in \textbf{regression models}. Note that the appropriate critical values depend on \textbf{sample size}. For any given level of significance, the critical values of the t-statistic decrease as sample size increases. 

\subsection{Augmented Dicker-Fuller test}
Consider the pth-order autoregressive process:
\begin{eqnarray*}
y_{t}&=&a_{0}+a_{1}y_{t-1}+a_{2}y_{t-2}+a_{3}y_{t-3}+\cdots+a_{p-2}y_{t-p+2}+a_{p-1}y_{t-p+1}+a_{p}y_{t-p}+u_{t}
\end{eqnarray*}
jj
Add and subtract $a_{p}y_{t-p+1}$
\begin{eqnarray*}
y_{t}&=&a_{0}+a_{1}y_{t-1}+a_{2}y_{t-2}+\cdots+a_{p-2}y_{t-p+2}+a_{p-1}y_{t-p+1}+a_{p}y_{t-p+1}+a_{p}y_{t-p}-a_{p}y_{t-p+1}+u_{t}\\
	&=&a_{0}+a_{1}y_{t-1}+a_{2}y_{t-2}+\cdots+a_{p-2}y_{t-p+2}+(a_{p-1}+a_{p})y_{t-p+1}-a_{p}\Delta y_{t-p+1}+u_{t}
\end{eqnarray*}

Add and subtract $(a_{p-1}+a_{p})y_{t-p+2}$
\begin{eqnarray*}
y_{t}&=&a_{0}+\cdots+a_{p-2}y_{t-p+2}+(a_{p-1}+a_{p})y_{t-p+2}+(a_{p-1}+a_{p})y_{t-p+1}-(a_{p-1}+a_{p})y_{t-p+2}-a_{p}\Delta y_{t-p+1}+u_{t}\\
	&=&a_{0}+\cdots+(a_{p-2}+a_{p-1}+a_{p})y_{t-p+2}-(a_{p-1}+a_{p})\Delta y_{t-p+2}-a_{p}\Delta y_{t-p+1}+u_{t}
\end{eqnarray*}

Continuing in this fashion, we get
\begin{eqnarray*}
\Delta y_{t}&=&a_{0}+\gamma y_{t-1}+\sum_{i=1}^{p}\beta_{i}\Delta y_{t-i+1}+u_{t}\\
where\ \ \gamma&=& -\left( 1-\sum_{i=1}^{p}a_{i}\right)\\
	\beta_{i}&=& \sum_{j=i}^{p}a_{j}
\end{eqnarray*}

We can use the same Dickey-Fuller statistics which depends on the regression models and sample size.
\end{document}