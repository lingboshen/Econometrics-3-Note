\documentclass{article}
\usepackage{comment}
\usepackage{amssymb}
\usepackage{mathrsfs}
\usepackage{amsmath,amsthm,amssymb,amsfonts}
\usepackage{amsfonts}
\usepackage{bm}
\usepackage{commath}
\usepackage{pgf,tikz}
\usetikzlibrary{arrows}
\pagestyle{empty}
\usepackage{indentfirst}
\usepackage[margin=3.5cm]{geometry}
\linespread{1.45}
\usepackage[colorlinks,linkcolor=red,anchorcolor=blue,citecolor=green]{hyperref}
\newtheorem{remark}{Remark}
\newtheorem{theorem}{Theorem}
\newtheorem{definition}{Definition}
\DeclareMathOperator*{\plim}{plim}

\newcommand{\tmop}[1]{\ensuremath{\operatorname{#1}}}
\newenvironment{enumeratenumeric}{\begin{enumerate}[1.] }{\end{enumerate}}
\newenvironment{enumerateroman}{\begin{enumerate}[i.] }{\end{enumerate}}
%\newenvironment{proof}{\noindent\textbf{Proof\ }}{\hspace*{\fill}$\Box$\medskip}
%%%%%%%%%% End TeXmacs macros

\newcommand{\x}{\bm{x}}


\newcommand*{\MyPath}{./Chapter}


\begin{document}




\title{Time Series Note}
\author{Lingbo Shen}
\maketitle

\tableofcontents
\newpage

%model
\section[Modelling Ecomomic Time Series]{Modelling Ecomomic Time Series}
\subsection{Data Generation Processes}
Let $\x_{t}$ be an $m\times 1$ vector of economic variables generated at time $t$. Such variables are typically inter0related both contemporaneously and across time. The collection $\{\x_{t},-\infty<t<\infty\}$ is called a vector-valued random sequence. 

The data generation process (DGP) is represented by the conditional density
\begin{eqnarray*}
D_{t}\left(\x_{t}\mid\mathcal{X}_{t-1}\right)
\end{eqnarray*}
where $\mathcal{X}_{t-1}=\sigma\left(\x_{t-1},\x_{t-2},\x_{t-3},\cdots\right)$. This is a shorthand for the $\sigma$-field representing knowledge of the past history of the system. Notice that
\begin{remark}
\item $D_{t}$ is allowed to depend on time, because the data are not assumed to be stationary.
\end{remark}

\subsection{DGPs and Models}
A dynamic econometric model is a family of functions of the data which are intended to mimic aspects of the DGP, either $D_{t}$ itself or functions derived from $D_{t}$ such as moments. Formally, a model is a family of functions
\begin{eqnarray*}
\{M\left(\x_{t},\x_{t-1},\x_{t-2},\cdots,\bm{d}_{t};\bm{\psi}\right),\bm{\psi}\in\Psi\},\Psi\subset\mathbb{R}^{p}
\end{eqnarray*}
In particular, let $M_{D}$ be a model of complete DGP.
\begin{itemize}
\item Parameters $\bm{\psi}$: $p$ in number; parameters are constants that are common to every $t$.
\item Parameter space: $\Psi$ denotes the set of admissible parameter values.
\item The vector $\bm{d}_{t}$ represents variables, treated as non-stochastic, which are intended to capture the changes in the DGP over time. 
\end{itemize}

The relationship between the DGP and the model is a difficult issue. The axiom of correct specification is the assumption that there exists a model element that is identical to the corresponding function of the DGP. $M_{D}$ is correctly specified if there exists $\bm{\psi}_{0}\in\Psi$ such that
\begin{eqnarray*}
M_{D}\left(\x_{t},\x_{t-1},\x_{t-2},\cdots,\bm{d}_{t};\bm{\psi}_{0}\right)&=&D_{t}\left(\x_{t}\mid\mathcal{X}_{t-1}\right)
\end{eqnarray*}
In general, correct specification in practical modelling exercises is an implausible assumption. 

\subsection{Sequence Properties}
\subsubsection{Stationarity}
A random sequence $\{\x_{t}\}$ is said to be \textit{stationary in the wide sense/covariance-stationary}
 if the mean, the variance and the sequence of $j$th-order autocovariances for $j>0$ are all independent of $t$. It is said to be \textit{stationary in the strict sense} if for every $k>0$, the joint distributions of all collections $\left(\x_{t},\x_{t+1},\x_{t+2},\cdots,\x_{t+k}\right)$ do not depend in any way on $t$.

\subsubsection{Mixing}
In a mixing sequence the realization of the sequence at time $t$ contains no information about the realization at either $t-j$ or $t+j$, when $j$ is sufficiently large. The mixing property ensures that points in the sequence appear randomly sampled when they are far enough apart.
\begin{remark}
Stationarity and mixing are quite distinct properties.
\end{remark}

\subsubsection{Ergodicity}
A stationary sequence having the property that a random event involving every member of the sequence always has probability either 0 or 1 is called \textit{ergodic}.
\begin{theorem}
If $\{x_{t}\}$ is a stationary ergodic sequence, and $E(x_1)$ exists, $\bar{x}_{n}\to E(x_{1})$ with probability 1.
\end{theorem}

%lag operator
\section[Difference Equation]{Difference Equation}
\subsection{The Difference Operator}
First differences
\begin{eqnarray*}
\Delta y_{t}&\equiv&y_{t}-y_{t-1}
\end{eqnarray*}

Second differences
\begin{eqnarray*}
\Delta^{2}y_{t}&\equiv&\Delta(\Delta y_{t})\\
			&\equiv&\Delta y_{t}-\Delta y_{t-1}\\
			&\equiv&(y_{t}-y_{t-1})-(y_{t-1}-y_{t-2})\\
			&\equiv&y_{t}-2y_{t-1}+y_{t-2}\\
\end{eqnarray*}

\subsection{Difference Equations and Solutions}
\begin{itemize}
\item n-th order linear difference equation (with constant coefficient)
\begin{eqnarray*}
y_{t}&=&a_{0}+\sum_{i=1}^{n}a_{i}y_{t-i}+X_{t}
\end{eqnarray*}
where $X_{t}$ is called forcing process. It includes stochastic terms, time trends, and other variables, but not constants and not y, nor lagged values of y.

\item difference form
\begin{eqnarray*}
y_{t}&=&a_{0}+\sum_{i=1}^{n}a_{i}y_{t-i}+X_{t}\\
y_{t}&=&a_{0}+a_{1}y_{t-1}+\sum_{i=2}^{n}a_{i}y_{t-i}+X_{t}\\
y_{t}-y_{t-1}&=&a_{0}+a_{1}y_{t-1}-y_{t-1}+\sum_{i=2}^{n}a_{i}y_{t-i}+X_{t}\\
\Delta y_{t}&=&a_{0}+(a_{1}-1)y_{t-1}+\sum_{i=2}^{n}a_{i}y_{t-i}+X_{t}\\
\end{eqnarray*}

\item A solution of a D.E shows $y_{t}$ equal to a function of the x and t, plus perhaps some initial conditions for y, but not lagged values of y. Solutions are not unique.
\end{itemize}

\subsubsection{Iteration}
Consider following first-order linear difference equation:
\begin{eqnarray*}
y_{t}&=&a_{0}+a_{1}y_{t-1}+\varepsilon_{t}, \ \ y_{0}=y_{0}
\end{eqnarray*}
We know 
\begin{eqnarray*}
y_{1}&=&a_{0}+a_{1}y_{0}+\varepsilon_{1}\\
y_{2}&=&a_{0}+a_{1}y_{1}+\varepsilon_{2}\\
	&=&a_{0}+a_{1}(a_{0}+a_{1}y_{0}+\varepsilon_{1})+\varepsilon_{2}\\
	&=&a_{0}+a_{1}a_{0}+a_{1}^{2}y_{0}+a_{1}\varepsilon_{1}+\varepsilon_{2}\\
y_{3}&=&a_{0}+a_{1}y_{2}+\varepsilon_{3}\\
	&=&a_{0}+a_{1}(a_{0}+a_{1}a_{0}+a_{1}^{2}y_{0}+a_{1}\varepsilon_{1}+\varepsilon_{2})+\varepsilon_{3}\\
	&=&a_{0}+a_{1}a_{0}+a_{1}^{2}a_{0}+a_{1}^{3}y_{0}+a_{1}^{2}\varepsilon_{1}+a_{1}\varepsilon_{2}+\varepsilon_{3}\\
	&\vdots&\\
y_{t}&=&a_{0}(1+a_{1}+a_{1}^{2}+\cdots+a_{1}^{t-1})+a_{1}^{t}y_{0}+a_{1}^{t-1}\varepsilon_{1}+a_{1}^{t-2}\varepsilon_{2}+\cdots+\varepsilon_{t}\\
&=&a_{0}\sum_{i=0}^{t-1}(a_{1})^{i}+a_{1}^{t}y_{0}+\sum_{i=0}^{t-1}(a_{1})^{i}\varepsilon_{t-i}
\end{eqnarray*}

\subsection{Lag Operators}
The lag operator L is defined 
\begin{eqnarray*}
L^{i}y_{t}\equiv y_{t-i}
\end{eqnarray*}
Lag operator has following properties
\begin{enumerate}
\item $L(\beta y_{t})=\beta\cdot Ly_{t}$
\item $L(x_{t}+y_{t})=Lx_{t}+Ly_{t}$
\item $(L^{i}+L^{j})y_{t}=L^{i}y_{t}+L^{j}y_{t}$
\item $L^{i}L^{j}y_{t}=L^{i+j}y_{t}=y_{t-i-j}$
\item $L^{-i}y_{t}=y_{t+i}$
\item For $\left |a\right|<1$
\begin{eqnarray*}
\sum_{i=0}^{+\infty}(a^{i}L^{i})y_{t}&=&\frac{y_{t}}{1-aL}
\end{eqnarray*}
\item For $\left |a\right|>1$
\begin{eqnarray*}
\sum_{i=0}^{+\infty}(a^{-i}L^{-i})y_{t}&=&\frac{-aLy_{t}}{1-aL}
\end{eqnarray*}
\end{enumerate}

\subsubsection{First-Order Difference Equations}
Consider the first-order difference equation below
\begin{eqnarray*}
y_{t}&=&\phi y_{t-1}+w_{t}
\end{eqnarray*}
which can be written using the lag operator as
\begin{eqnarray*}
y_{t}&=&\phi Ly_{t}+w_{t}\\
y_{t}-\phi Ly_{t}&=&w_{t}\\
(1-\phi L)y_{t}&=&w_{t}\\
(1+\phi L+\phi^{2}L^{2}+\cdots+\phi^{t}L^{t})(1-\phi L)y_{t}&=&(1+\phi L+\phi^{2}L^{2}+\cdots+\phi^{t}L^{t})w_{t}\\
\end{eqnarray*}
The compound operator on the left-hand side is
\begin{eqnarray*}
&&(1+\phi L+\phi^{2}L^{2}+\cdots+\phi^{t}L^{t})(1-\phi L)\\
&=&(1+\phi L+\phi^{2}L^{2}+\cdots+\phi^{t}L^{t})\\
&-&(\phi L+\phi^{2}L^{2}+\cdots+\phi^{t}L^{t}+\phi^{t+1}L^{t+1})\\
&=&1-\phi^{t+1}L^{t+1}
\end{eqnarray*}
Then we have
\begin{eqnarray*}
(1-\phi^{t+1}L^{t+1})y_{t}&=&(1+\phi L+\phi^{2}L^{2}+\cdots+\phi^{t}L^{t})w_{t}\\
y_{t}-\phi^{t+1}y_{-1}&=&w_{t}+\phi w_{t-1}+\phi^{2}w_{t-2}+\cdots+\phi^{t}w_{0}\\
y_{t}&=&\phi^{t+1}y_{-1}+w_{t}+\phi w_{t-1}+\phi^{2}w_{t-2}+\cdots+\phi^{t}w_{0}
\end{eqnarray*}

We have known that
\begin{eqnarray*}
(1+\phi L+\phi^{2}L^{2}+\cdots+\phi^{t}L^{t})(1-\phi L)y_{t}&=&y_{t}-\phi^{t+1}y_{-1}
\end{eqnarray*}
if $|\phi|<1$ and $y_{-1}$ is finite, $\phi^{t+1}y_{-1}$ will become negligible as $t$ becomes large
\begin{eqnarray*}
(1+\phi L+\phi^{2}L^{2}+\cdots+\phi^{t}L^{t})(1-\phi L)y_{t}&\cong&y_{t}
\end{eqnarray*}
A sequence $\{y_{t}\}_{t=-\infty}^{\infty}$ is bounded if there exists a finite number $\bar{y}$ such that $|y_{t}|<\bar{y}$ for all $t$. When $|\phi|<1$ and when we are considering applying an operator to a bounded sequence, we have
\begin{eqnarray*}
(1-\phi L)^{-1}&=&\lim_{t\to\infty} (1+\phi L+\phi^{2}L^{2}+\cdots+\phi^{t}L^{t})
\end{eqnarray*}
So we have
\begin{eqnarray*}
(1+\phi L+\phi^{2}L^{2}+\cdots+\phi^{t}L^{t})(1-\phi L)y_{t}&\cong&y_{t}\\
(1-\phi L)^{-1}(1-\phi L)y_{t}&=&y_{t}
\end{eqnarray*}
By this definition, we have
\begin{eqnarray*}
(1-\phi L)y_{t}&=&w_{t}\\
y_{t}&=&(1-\phi L)^{-1}w_{t}\\
&=&w_{t}+\phi w_{-1}+\phi^{2}w_{t-2}+\phi^{3}w_{t-3}+\cdots
\end{eqnarray*}

\subsubsection{Second-Order Difference Equations}
Consider the second-order difference equation
\begin{eqnarray*}
y_{t}&=&\phi_{1}y_{t-1}+\phi_{2}y_{t-2}+w_{t}\\
(1-\phi_{1}L-\phi_{2}L^{2})y_{t}&=&w_{t}
\end{eqnarray*}
The operator in the left-hand side contains a second-order polynomial in the lag operator $L$. Suppose we have
\begin{eqnarray*}
(1-\phi_{1}L-\phi_{2}L^{2})&=&(1-\lambda_{1}L)(1-\lambda_{2}L)\\
&=&(1-[\lambda_{1}+\lambda_{2}]L+\lambda_{1}\lambda_{2}L^{2})
\end{eqnarray*}
Given values for $\phi_{1}$ and $\phi_{2}$, we seek numbers $\lambda_{1}$ and $\lambda_{2}$ such that
\begin{eqnarray*}
\begin{cases}
\lambda_{1}+\lambda_{2}=\phi_{1}\\
\lambda_{1}\lambda_{2}=-\phi_{2}
\end{cases}
\end{eqnarray*}
Two methods to consider this problem.
\begin{itemize}
\item Method 1
\begin{eqnarray*}
(1-\phi_{1}z-\phi_{2}z^{2})&=&(1-\lambda_{1}z)(1-\lambda_{2}z)
\end{eqnarray*}
when $z=\lambda_{1}^{-1}$ or $z=\lambda_{2}^{-1}$, the right hand side is equal to 0. When
\begin{eqnarray*}
\begin{cases}
z_{1}=\frac{\phi_{1}-\sqrt{\phi_{1}^{2}+4\phi_{2}}}{-2\phi_{2}}\\
\\
z_{2}=\frac{\phi_{1}+\sqrt{\phi_{1}^{2}+4\phi_{2}}}{-2\phi_{2}}
\end{cases}
\end{eqnarray*}
the left-hand side is equal to 0 as well. So if we let 
\begin{eqnarray*}
\begin{cases}
\lambda_{1}^{-1}=z_{1}=\frac{\phi_{1}-\sqrt{\phi_{1}^{2}+4\phi_{2}}}{-2\phi_{2}}\\
\\
\lambda_{2}^{-1}=z_{2}=\frac{\phi_{1}+\sqrt{\phi_{1}^{2}+4\phi_{2}}}{-2\phi_{2}}
\end{cases}
\end{eqnarray*}
both sides are equal to 0.
\item Method 2\\
It is easy to find that $\lambda_{1}$ and $\lambda_{2}$ are roots of equation 
\begin{eqnarray*}
\lambda^{2}-\phi_{1}\lambda-\phi_{2}&=&0
\end{eqnarray*}
so 
\begin{eqnarray*}
\begin{cases}
\lambda_{1}^{-1}=\frac{\phi_{1}+\sqrt{\phi_{1}^{2}+4\phi_{2}}}{2}\\
\\
\lambda_{2}^{-1}=\frac{\phi_{1}-\sqrt{\phi_{1}^{2}+4\phi_{2}}}{2}
\end{cases}
\end{eqnarray*}
\end{itemize}

When we have $\lambda_{1}$ and $\lambda_{2}$ by the above two methods, we have 
\begin{eqnarray*}
(1-\lambda_{1}L)(1-\lambda_{2}L)y_{t}&=&w_{t}\\
y_{t}&=&(1-\lambda_{1}L)^{-1}(1-\lambda_{2}L)^{-1}w_{t}
\end{eqnarray*}

If $\lambda_{1}\neq \lambda_{2}$, we define
\begin{eqnarray*}
&&(\lambda_{1}-\lambda_{2})^{-1}\left(\frac{\lambda_{1}}{1-\lambda_{1}L}-\frac{\lambda_{2}}{1-\lambda_{2}L}\right)\\
&=&(\lambda_{1}-\lambda_{2})^{-1}\frac{\lambda_{1}(1-\lambda_{2}L)-\lambda_{2}(1-\lambda_{1}L)}{(1-\lambda_{1}L)(1-\lambda_{2}L)}\\
&=&(\lambda_{1}-\lambda_{2})^{-1}\frac{\lambda_{1}-\lambda_{2}}{(1-\lambda_{1}L)(1-\lambda_{2}L)}\\
&=&\frac{1}{(1-\lambda_{1}L)(1-\lambda_{2}L)}
\end{eqnarray*}
So we have
\begin{eqnarray*}
y_{t}&=&(1-\lambda_{1}L)^{-1}(1-\lambda_{2}L)^{-1}w_{t}\\
&=&(\lambda_{1}-\lambda_{2})^{-1}\left(\frac{\lambda_{1}}{1-\lambda_{1}L}-\frac{\lambda_{2}}{1-\lambda_{2}L}\right)w_{t}\\
&=&\frac{\lambda_{1}}{\lambda_{1}-\lambda_{2}}\frac{1}{1-\lambda_{1}L}w_{t}-\frac{\lambda_{2}}{\lambda_{1}-\lambda_{2}}\frac{1}{1-\lambda_{2}L}w_{t}\\
&=&\frac{\lambda_{1}}{\lambda_{1}-\lambda_{2}}(1+\lambda_{1}L+\lambda_{1}^{2}L^{2}+\lambda_{1}^{3}L^{3}+\cdots)w_{t}\\
&-&\frac{\lambda_{2}}{\lambda_{1}-\lambda_{2}}(1+\lambda_{2}L+\lambda_{2}^{2}L^{2}+\lambda_{2}^{3}L^{3}+\cdots)w_{t}\\
&=&\frac{\lambda_{1}}{\lambda_{1}-\lambda_{2}}(w_{t}+\lambda_{1}w_{t-1}+\lambda_{1}^{2}w_{t-2}+\lambda_{1}^{3}w_{t-3}+\cdots)\\
&-&\frac{\lambda_{2}}{\lambda_{1}-\lambda_{2}}(w_{t}+\lambda_{2}w_{t-1}+\lambda_{2}^{2}w_{t-2}+\lambda_{2}^{3}w_{t-3}+\cdots)\\
&=&(c_{1}+c_{2})w_{t}+(c_{1}\lambda_{1}+c_{2}\lambda_{2})w_{t-1}+(c_{1}\lambda_{1}^{2}+c_{2}\lambda_{2}^{2})w_{t-2}+\cdots
\end{eqnarray*}
where $c_{1}=\lambda_{1}/(\lambda_{1}-\lambda_{2})$ and $c_{2}=-\lambda_{2}/(\lambda_{1}-\lambda_{2})$.

\subsubsection{$p$th-Order Difference Equations}

%arma
\section[ARMA]{ARMA}
The autoregressive moving average (ARMA) model
\begin{eqnarray*}
y_{t}&=&a_{0}+\sum_{i=1}^{p}a_{i}y_{t-i}+\sum_{i=0}^{q}\beta_{i}\varepsilon_{t-i}
\end{eqnarray*}

\subsection{Stationarity}
A stochastic process is \textit{covariance stationary} or \textit{weakly stationary} if for all t and s
\begin{eqnarray*}
E(y_{t})&=&E(y_{t-s})=\mu\\
E\left[(y_{t}-\mu)^{2}\right]&=&E\left[(y_{t-s}-\mu)^{2}\right]=\sigma_{y}^{2}=\gamma_{0}\\
E\left[(y_{t}-\mu)(y_{t-s}-\mu)\right]&=&E\left[(y_{t-j}-\mu)(y_{t-j-s}-\mu)\right]=\gamma_{s}
\end{eqnarray*}
If a process is covariance stationary, the covariance between $y_{t}$ and $y_{t-s}$ depends only on $s$, the length of time separating the observations. It follows that for a covariance stationary process, $\gamma_{s}$ and $\gamma_{-s}$ would represent the same magnitude. 


For a covariance stationary series, we can define the autocorrelation between $y_{t}$ and $y_{t-s}$
\begin{eqnarray*}
\rho_{s}&=&\frac{\gamma_{s}}{\gamma_{0}}
\end{eqnarray*}
where $\gamma_{0}$ is the variance of $y_{t}$

\subsection{Ergodicity}
Imagine a battery of $I$ computers generating sequences $\{y_{t}^{(1)}\}_{t=-\infty}^{\infty}$, $\{y_{t}^{(2)}\}_{t=-\infty}^{\infty}$, $\dots$, $\{y_{t}^{(I)}\}_{t=-\infty}^{\infty}$ and consider selecting the observation associated with date $t$ from each sequence:
$$\{y_{t}^{(1)},y_{t}^{(2)},\dots,y_{t}^{(I)}\}$$
This would be described as a sample of $I$ realizations of the random variable $Y_{t}$. The expectation of the $t$th observation of a time series refers to the mean of the probability distribution
\begin{eqnarray*}
E(Y_{t})&=&\int_{-\infty}^{\infty}y_{t}f_{Y_{t}}(y_{t})dy_{t}
\end{eqnarray*}
We might view this as the probability limit of the ensemble average
\begin{eqnarray*}
E(Y_{t})&=&\plim_{I\to\infty}\frac{1}{I}\sum_{i=1}^{I}Y_{t}^{(i)}
\end{eqnarray*}

The above expectations of a time series in terms of ensemble averages may seem a bit contrived. Usually we have a single realization of size $T$ from the process 
$$\{y_{1}^{(1)},y_{2}^{(1)},\dots,y_{T}^{(1)}\}$$
From these observations we would calculate the sample mean $\bar{y}$, which is a time average
\begin{eqnarray*}
\bar{y}&=&\frac{1}{T}\sum_{t=1}^{T}y_{t}^{(1)}
\end{eqnarray*}

A covariance stationary process is said to be \textit{ergodic for the mean} if $\bar{y}$ converges in probability to $E(Y_{t})$ as $T\to\infty$. 

\subsection{Moving Average Processes}
\subsubsection{The First-Order Moving Average Process}
Let $\{\varepsilon_{t}\}$ be white noise and consider the process
\begin{eqnarray*}
y_{t}&=&\mu+\varepsilon_{t}+\theta\varepsilon_{t-1}
\end{eqnarray*}
where $\mu$ and $\theta$ could be any constants. This time series is called a first-order moving average process, denoted $MA(1)$.

\paragraph{Expectation}
The expectation of $y_{t}$ is 
\begin{eqnarray*}
E(y_{t})&=&E(\mu+\varepsilon_{t}+\theta\varepsilon_{t-1})=\mu+E(\varepsilon)+\theta E(\varepsilon_{t-1})=\mu
\end{eqnarray*}

\paragraph{Variance}
The variance of $y_{t}$ is 
\begin{eqnarray*}
E(y_{t}-\mu)^{2}&=&E(\varepsilon_{t}+\theta\varepsilon_{t-1})^{2}\\
&=&E\left(\varepsilon_{t}^{2}+2\theta\varepsilon_{t}\varepsilon_{t-1}+\theta^{2}\varepsilon_{t-1}^{2}\right)\\
&=&\sigma^{2}+0+\theta^{2}\sigma^{2}\\
&=&(1+\theta^{2})\sigma^{2}
\end{eqnarray*}

\paragraph{Autocovariance}
The first autocovariance of $y_{t}$ is 
\begin{eqnarray*}
E(y_{t}-\mu)(y_{t-1}-\mu)&=&E(\varepsilon_{t}+\theta\varepsilon_{t-1})(\varepsilon_{t-1}+\theta\varepsilon_{t-2})\\
&=&E\left(\varepsilon_{t}\varepsilon_{t-1}+\theta\varepsilon_{t-1}^{2}+\theta\varepsilon_{t}\varepsilon_{t-2}+\theta^{2}\varepsilon_{t-1}\varepsilon_{t-2}\right)\\
&=&0+\theta\sigma^{2}+0+0\\
&=&\theta^{2}\sigma^{2}
\end{eqnarray*}

Higher autocovariances are all zero. For all $j>1$
\begin{eqnarray*}
E(y_{t}-\mu)(y_{t-j}-\mu)&=&0
\end{eqnarray*}

\paragraph{Autocorrelation}
The $j$th autocorrelation of a covariance stationary process is $\rho_{j}=\frac{\gamma_{j}}{\gamma_{0}}$
\begin{eqnarray*}
\rho_{1}&=&\frac{\gamma_{1}}{\gamma_{0}}=\frac{\theta^{2}\sigma^{2}}{(1+\theta^{2})\sigma^{2}}=\frac{\theta^{2}}{1+\theta^{2}}\\
\rho_{j}&=&0,\ \ \text{$j>1$}
\end{eqnarray*}


\subsubsection{The $q$th-Order Moving Average Process}
A $q$th-order moving average process, denoted $MA(q)$, is characterized by
\begin{eqnarray*}
y_{t}&=&\mu+\varepsilon_{t}+\theta_{1}\varepsilon_{t-1}+\theta_{2}\varepsilon_{t-2}+\cdots+\theta_{q}\varepsilon_{t-q}
\end{eqnarray*}

\paragraph{Expectation}
The expectation of $y_{t}$ is 
\begin{eqnarray*}
E(y_{t})&=&E(\mu)+E(\varepsilon_{t})+\theta_{1}E(\varepsilon_{t-1})+\theta_{2}E(\varepsilon_{t-2})+\cdots+\theta_{q}E(\varepsilon_{t-q})=\mu
\end{eqnarray*}

\paragraph{Variance}
The variance of $y_{t}$ is 
\begin{eqnarray*}
E(y_{t}-\mu)^{2}&=&E(\varepsilon_{t}+\theta_{1}\varepsilon_{t-1}+\theta_{2}\varepsilon_{t-2}+\cdots+\theta_{q}\varepsilon_{t-q})^{2}\\
&=&E(\varepsilon_{t})^{2}+E(\theta_{1}\varepsilon_{t-1})^{2}+E(\theta_{2}\varepsilon_{t-2})^{2}+\cdots+E(\theta_{q}\varepsilon_{t-q})^{2}\\
&=&\sigma^{2}+\theta_{1}^{2}\sigma^{2}+\theta_{2}^{2}\sigma^{2}+\cdot+\theta_{q}^{2}\sigma^{2}\\
&=&(1+\theta_{1}^{2}+\theta_{2}^{2}+\cdot+\theta_{q}^{2})\sigma^{2}
\end{eqnarray*}

\paragraph{Autocovariance}
The autocovariance of $y_{t}$ is 
\begin{eqnarray*}
E(y_{t}-\mu)(y_{t-j}-\mu)&=&E(\varepsilon_{t}+\theta_{1}\varepsilon_{t-1}+\theta_{2}\varepsilon_{t-2}+\cdots+\theta_{q}\varepsilon_{t-q})\\
&&(\varepsilon_{t-j}+\theta_{1}\varepsilon_{t-j-1}+\theta_{2}\varepsilon_{t-j-2}+\cdots+\theta_{q}\varepsilon_{t-j-q})\\
&=&E\left(\theta_{j}\varepsilon_{t-j}^{2}+\theta_{j+1}\theta_{1}\varepsilon_{t-j-1}^{2}+\theta_{j+2}\theta_{2}\varepsilon_{t-j-2}^{2}+\cdots+\theta_{q}\theta_{q-j}\varepsilon_{t-q}^{2}\right)\\
&=&\left(\theta_{j}+\theta_{j+1}\theta_{1}+\theta_{j+2}\theta_{2}+\cdots+\theta_{q}\theta_{q-j}\right)\sigma^{2}, \ \ \text{$j=1,2,\dots,q$}
\end{eqnarray*}

For all $j>q$
\begin{eqnarray*}
E(y_{t}-\mu)(y_{t-j}-\mu)&=&0
\end{eqnarray*}

\subsubsection{The Infinite-Order Moving Average Process}
Consider the process when $q\to\infty$ 
\begin{eqnarray*}
y_{t}&=&\mu+\sum_{j=0}^{\infty}\psi_{j}\varepsilon_{t-j}=\mu+\psi_{0}\varepsilon_{t}+\psi_{1}\varepsilon_{t-1}+\psi_{2}\varepsilon_{t-2}+\cdots
\end{eqnarray*}
This could be described as an $MA(\infty)$ process.

The $MA(\infty)$ process is covariance stationary if it is square summable
\begin{eqnarray*}
\sum_{j=0}^{\infty}\psi_{j}^{2}&<&\infty
\end{eqnarray*}
It is often to work with a slightly stronger condition called absolutely summable
\begin{eqnarray*}
\sum_{j=0}^{\infty}|\psi_{j}|&<&\infty
\end{eqnarray*}

\paragraph{Expectation}
The mean of an $MA(\infty)$ process with absolutely summable is
\begin{eqnarray*}
E(y_{t})&=&\lim_{T\infty}E(\mu+\psi_{0}\varepsilon_{t}+\psi_{1}\varepsilon_{t-1}+\psi_{2}\varepsilon_{t-2}+\cdots+\psi_{T}\varepsilon_{t-T})=\mu
\end{eqnarray*}

\paragraph{Autocovariance}
The autocovariance of an $MA(\infty)$ process with absolutely summable is
\begin{eqnarray*}
\gamma_{0}&=&E(y_{t}-\mu)^{2}\\
&=&\lim_{T\infty}E(\psi_{0}\varepsilon_{t}+\psi_{1}\varepsilon_{t-1}+\psi_{2}\varepsilon_{t-2}+\cdots+\psi_{T}\varepsilon_{t-T})^{2}\\
&=&\lim_{T\to\infty}(\psi_{0}^{2}+\psi_{1}^{2}+\psi_{2}^{2}+\cdots+\psi_{T}^{2})\sigma^{2}\\
\gamma_{j}&=&E(y_{t}-\mu)(y_{t-j}-\mu)\\
&=&(\psi_{j}\psi_{0}+\psi_{j+1}\psi_{1}+\psi_{j+2}\psi_{2}+\psi_{j+3}\psi_{3}+\cdots)\sigma^{2}
\end{eqnarray*}

\subsection{Autoregressive Processes}
\subsubsection{The First-Order Autoregressive Process}
A first-order autoregressive, denoted $AR(1)$, satisfies the following difference equation
\begin{eqnarray*}
y_{t}&=&c+\phi y_{t-1}+\varepsilon_{t}
\end{eqnarray*}
When $|\phi|<1$, this process is covariance stationary. The $AR(1)$ process can be rewritten as $MA(\infty)$ process as follows
\begin{eqnarray*}
y_{t}&=&c+\phi y_{t-1}+\varepsilon_{t}\\
&=&c+\varepsilon_{t}+\phi (c+\varepsilon_{t-1})+\phi^{2} y_{t-2}\\
&=&c+\varepsilon_{t}+\phi (c+\varepsilon_{t-1})+\phi^{2} (c+\varepsilon_{t-2})+\phi^{3} y_{t-3}\\
&=&c+\varepsilon_{t}+\phi (c+\varepsilon_{t-1})+\phi^{2} (c+\varepsilon_{t-2})+\phi^{3} (c+\varepsilon_{t-3})+\phi^{4} y_{t-4}\\
&=&c+\varepsilon_{t}+\phi (c+\varepsilon_{t-1})+\phi^{2} (c+\varepsilon_{t-2})+\phi^{3} (c+\varepsilon_{t-3})+\cdots\\
&=&(c+\phi c+\phi^{2}c+\phi^{3}c+\cdots)+\varepsilon_{t}+\phi\varepsilon_{t-1}+\phi^{2}\varepsilon_{t-2}+\phi^{3}\varepsilon_{t-3}+\cdots\\
&=&\frac{c}{1-\phi}+\varepsilon_{t}+\phi\varepsilon_{t-1}+\phi^{2}\varepsilon_{t-2}+\phi^{3}\varepsilon_{t-3}+\cdots
\end{eqnarray*}

We can derive the expectation and autocovariance of $AR(1)$ by the above corresponding $MA(\infty)$ process. We also can derive them by assuming $AR(1)$ process is covariance stationary.

\paragraph{Expectation}
Taking expectations both sides
\begin{eqnarray*}
E(y_{t})&=&c+\phi E(y_{t-1})+E(\varepsilon_{t})\\
\mu&=&c+\phi\mu\\
\mu&=&\frac{c}{1-\phi}
\end{eqnarray*}
\paragraph{Autocovariance}



\subsubsection{The $q$th-Order Autoregressive Process}

\subsection{The Autocorrelation Function}
Autocorrelation function (ACF) and the partial autocorrelation function (PACF) are useful to determine the type of time series data.

For AR(1) model
\begin{itemize}
\item Method 1
\begin{eqnarray*}
y_{t}&=&a_{0}+a_{1}y_{t-1}+\varepsilon_{t}\ \ assume \ stationary\\
E(y_{t})&=&a_{0}+a_{1}E(y_{t-1})+E(\varepsilon_{t})\\
\Rightarrow\mu&=&a_{0}+a_{1}\mu\\
\Rightarrow\mu&=&\frac{a_{0}}{1-a_{1}}\\
Var(y_{t})&=&a_{1}^{2}Var(y_{t-1})+Var(\varepsilon_{t})\\
\Rightarrow \gamma_{0}&=&a_{1}^{2}\gamma_{0}+\sigma^{2}\\
\Rightarrow \gamma_{0}&=&\frac{\sigma^{2}}{1-a_{1}^{2}}\\
Cov(y_{t}, y_{t-s})&=&Cov(a_{0}+a_{1}y_{t-1}+\varepsilon_{t}, a_{0}+a_{1}y_{t-s-1}+\varepsilon_{t-s})\\
			    &=&Cov(a_{1}y_{t-1}+\varepsilon_{t}, a_{1}y_{t-s-1}+\varepsilon_{t-s})\\
\Rightarrow \gamma_{s}&=&a_{1}^{2}\gamma_{s}+a_{1}^{s}\sigma^{2}\\
\Rightarrow \gamma_{s}&=&\frac{a_{1}^{s}\sigma^{2}}{1-a_{1}^{2}}\\
\end{eqnarray*}
So the autocorrelation function for AR(1)
\begin{eqnarray*}
\rho_{s}&=&\frac{\gamma_{s}}{\gamma_{0}}\\
&=&a_{1}^{s}
\end{eqnarray*}

\item Method 2 \\
If the process started at time zero
\begin{eqnarray*}
y_{t}&=&a_{0}\sum_{i=0}^{t-1}a_{1}^{i}+a_{1}^{t}y_{0}+\sum_{i=0}^{t-1}a_{1}^{i}\varepsilon_{t-i}
\end{eqnarray*}
Take the expectation of $y_{t}$ and $y_{t+s}$
\begin{eqnarray*}
E(y_{t})&=&a_{0}\sum_{i=0}^{t-1}a_{1}^{i}+a_{1}^{t}y_{0}\\
E(y_{t+s})&=&a_{0}\sum_{i=0}^{t+s-1}a_{1}^{i}+a_{1}^{t+s}y_{0}
\end{eqnarray*}
If $|a_{1}|<1$, as $t\rightarrow \infty$
\begin{eqnarray*}
\lim_{t\rightarrow \infty}y_{t}&=&\frac{a_{0}}{1-a_{1}}+\sum_{i=0}^{\infty}a_{1}^{i}\varepsilon_{t-i}
\end{eqnarray*}
\begin{eqnarray*}
Var(y_{t})&=&E\left[ (y_{t}-\mu)^{2}\right]=E\left[ (\varepsilon_{t}+a_{1}\varepsilon_{t-1}+a_{1}^{2}\varepsilon_{t-2}+\cdots)^{2}\right]\\
&=&\sigma^{2}\left[ (1+a_{1}^{2}+a_{1}^{4}+\cdots)\right]=\frac{\sigma^{2}}{1-a_{1}^{2}}
\end{eqnarray*}
\begin{eqnarray*}
Cov(y_{t}, y_{t-s})&=&E\left[ (y_{t}-\mu)(y_{t-s}-\mu)\right]\\
&=&E\left[ (\varepsilon_{t}+a_{1}\varepsilon_{t-1}+a_{1}^{2}\varepsilon_{t-2}+\cdots)(\varepsilon_{t-s}+a_{1}\varepsilon_{t-s-1}+a_{1}^{2}\varepsilon_{t-s-2}+\cdots)\right]\\
&=&E\left( a_{1}^{s}\varepsilon_{t-s}^{2}+a_{1}^{s+2}\varepsilon_{t-s-1}^{2}+a_{1}^{s+4}\varepsilon_{t-s-2}^{2}+\cdots\right)\\
&=&\sigma^{2}a_{1}^{s}(1+a_{1}^{2}+a_{1}^{4}+\cdots)\\
&=&\frac{\sigma^{2}a_{1}^{s}}{1-a_{1}^{2}}
\end{eqnarray*}
\end{itemize}

\subsubsection{The Autocorrelation Function of an AR(2) Process}
We assume that $a_{0}=0$, which implies that $E(y_{t})=0$. Adding or subtracting any constant from a variable does not change its variance, covariance, correlation coefficient, etc.

Using Yule-Walker equations: multiply the second-order D.E by $y_{t-s}$ for s$=0, 1, 2, \cdots$, and take expectations
\begin{eqnarray*}
Ey_{t}y_{t}&=&a_{1}Ey_{t-1}y_{t}+a_{2}Ey_{t-2}y_{t}+E\varepsilon_{t}y_{t}\\
Ey_{t}y_{t-1}&=&a_{1}Ey_{t-1}y_{t-1}+a_{2}Ey_{t-2}y_{t-1}+E\varepsilon_{t}y_{t-1}\\
Ey_{t}y_{t-2}&=&a_{1}Ey_{t-1}y_{t-2}+a_{2}Ey_{t-2}y_{t-2}+E\varepsilon_{t}y_{t-2}\\
&\vdots&\\
Ey_{t}y_{t-s}&=&a_{1}Ey_{t-1}y_{t-s}+a_{2}Ey_{t-2}y_{t-s}+E\varepsilon_{t}y_{t-s}\\
\end{eqnarray*}
By definition, the autocovariances of a stationary series are such 
\begin{eqnarray*}
Ey_{t}y_{t-s}&=&Ey_{t-s}y_{t}=Ey_{t-k}y_{t-k-s}=\gamma_{s}
\end{eqnarray*}
We also know that coefficient on $\varepsilon_{t}$ is unity so that $E\varepsilon_{t}y_{t}=\sigma^{2}$, and $E\varepsilon_{t}y_{t-s}=0$, so
\begin{eqnarray*}
\gamma_{0}&=&a_{1}\gamma_{1}+a_{2}\gamma_{2}+\sigma^{2}\\
\gamma_{1}&=&a_{1}\gamma_{0}+a_{2}\gamma_{1}\\
\gamma_{2}&=&a_{1}\gamma_{1}+a_{2}\gamma_{0}\\
&\vdots&\\
\gamma_{s}&=&a_{1}\gamma_{s-1}+a_{2}\gamma_{s-2}
\end{eqnarray*}
Now we can get the ACF
\begin{eqnarray*}
\rho_{1}&=&a_{1}\rho_{0}+a_{2}\rho_{1}\\
\rho_{s}&=&a_{1}\rho_{s-1}+a_{2}\rho_{s-2}
\end{eqnarray*}
We know $\rho_{0}=1$, so
\begin{eqnarray*}
\rho_{1}&=&a_{1}+a_{2}\rho_{1}\\
\rho_{1}&=&\frac{a_{1}}{1-a_{2}}
\end{eqnarray*}

\subsubsection{The Autocorrelation Function of an MA(1) Process}
Consider the MA(1) process $y_{t}=\varepsilon_{t}+\beta\varepsilon_{t-1}$

Applying the Yule-Walker equations 
\begin{eqnarray*}
\gamma_{0}&=&E(y_{t}y_{t})=E\left[ (\varepsilon_{t}+\beta\varepsilon_{t-1})(\varepsilon_{t}+\beta\varepsilon_{t-1})\right]=(1+\beta^{2})\sigma^{2}\\
\gamma_{1}&=&E(y_{t}y_{t-1})=E\left[ (\varepsilon_{t}+\beta\varepsilon_{t-1})(\varepsilon_{t-1}+\beta\varepsilon_{t-2})\right]=\beta^{2}\sigma^{2}\\
\gamma_{2}&=&E(y_{t}y_{t-2})=E\left[ (\varepsilon_{t}+\beta\varepsilon_{t-1})(\varepsilon_{t-2}+\beta\varepsilon_{t-3})\right]=0\\
\gamma_{s}&=&E(y_{t}y_{t-s})=E\left[ (\varepsilon_{t}+\beta\varepsilon_{t-1})(\varepsilon_{t-s}+\beta\varepsilon_{t-s-1})\right]=0 \ \ \forall t>2
\end{eqnarray*}

So the ACF of MA(1)
\begin{eqnarray*}
\rho_{0}&=&1\\
\rho_{1}&=&\frac{\gamma_{1}}{\gamma_{0}}=\frac{\beta^{2}}{1+\beta^{2}}\\
\rho_{s}&=&0 \ \ \forall t>1
\end{eqnarray*}

\subsubsection{The Autocorrelation Function of an ARMA(1,1) Process}
Consider the ARMA(1,1) $y_{t}=a_{1}y_{t-1}+\varepsilon_{t}+\beta_{1}\varepsilon_{t-1}$
\begin{eqnarray*}
Ey_{t}y_{t}&=&a_{1}Ey_{t-1}y_{t}+E\varepsilon_{t}y_{t}+\beta_{1}E\varepsilon_{t-1}y_{t} \Rightarrow \\
\gamma_{0}&=&a_{1}\gamma_{1}+\sigma^{2}+\beta_{1}(a_{1}+\beta_{1})\sigma^{2}\\
Ey_{t}y_{t-1}&=&a_{1}Ey_{t-1}y_{t-1}+E\varepsilon_{t}y_{t-1}+\beta_{1}E\varepsilon_{t-1}y_{t-1} \Rightarrow \\
\gamma_{1}&=&a_{1}\gamma_{0}+\beta_{1}\sigma^{2}\\
Ey_{t}y_{t-2}&=&a_{1}Ey_{t-1}y_{t-2}+E\varepsilon_{t}y_{t-2}+\beta_{1}E\varepsilon_{t-1}y_{t-2} \Rightarrow \\
\gamma_{2}&=&a_{1}\gamma_{1}\\
Ey_{t}y_{t-s}&=&a_{1}Ey_{t-1}y_{t-s}+E\varepsilon_{t}y_{t-s}+\beta_{1}E\varepsilon_{t-1}y_{t-s} \Rightarrow \\
\gamma_{s}&=&a_{1}\gamma_{s-1}
\end{eqnarray*}
Solve the equations and get 
\begin{eqnarray*}
\gamma_{0}&=&\frac{1+\beta_{1}^{2}+2a_{1}\beta_{1}}{1-a_{1}^{2}}\sigma^{2}\\
\gamma_{1}&=&\frac{(1+a_{1}\beta_{1})(a_{1}+\beta_{1})}{1-a_{1}^{2}}\sigma^{2}
\end{eqnarray*}
And the AFC 
\begin{eqnarray*}
\rho_{0}&=&1\\
\rho_{1}&=&\frac{\gamma_{1}}{\gamma_{0}}=\frac{(1+a_{1}\beta_{1})(a_{1}+\beta_{1})}{1+\beta_{1}^{2}+2a_{1}\beta_{1}}\\
\rho_{s}&=&a_{1}\rho_{s} \ \ \forall t>1
\end{eqnarray*}

\subsection{The Partial Autoorrelation Function}
In AR(1) process, $y_{t}$ and $y_{t-2}$ are correlated even though $y_{t-2}$ does not directly appear in the model. $\rho_{2}=corr(y_{t}, y_{t-1})\times corr(y_{t-1}, y_{t-2})=\rho_{1}^{2}$. All such "indirect" correlations are present in the ACF. In contrast, the partial autocorrelation between $y_{t}$ and $y_{t-s}$ climinates the effects of the intervening values $y_{t-1}$ through $y_{t-s+1}$.

Method to find the PACF:
\begin{enumerate}
\item Form the series $\{y_{t}^{\ast}\}$, where $y_{t}^{\ast}\equiv y_{t}-\mu$
\item Form the first-order autoregression equation:
	\begin{eqnarray*}
	y_{t}^{\ast}&=&\phi_{11}y_{t-1}^{\ast}+e_{i}\\
	y_{t}^{\ast}&=&\phi_{21}y_{t-1}^{\ast}+\phi_{22}y_{t-2}^{\ast}+e_{i}
	\end{eqnarray*}

where $\phi_{11}$ is the partial autocorrelation between $y_{t}$ and $y_{t-1}$, $\phi_{22}$ is the partial autocorrelation between $y_{t}$ and $y_{t-2}$. Repeating this process for all additional lags s yields the PACF.
\end{enumerate}

\begin{table}[!h]
\caption{Properties of the ACF and PACF}
\label{acf}
\begin{center}
\begin{tabular}{lll}\hline
Process		&		ACF		&PACF\\ \hline
White-noise& 	$\rho_{s}=0$&$\phi_{ss}=0$\\
AR(1): $a_{1}>0$&$\rho_{s}=a_{1}^{s}$ 	&$\phi_{11}=\rho_{1}$; $\phi_{ss}=0$ for $s\geq2$\\
AR(1): $a_{1}<0$&$\rho_{s}=a_{1}^{s}$ 	&$\phi_{11}=\rho_{1}$; $\phi_{ss}=0$ for $s\geq2$\\
AR(p)&Decays toward zero.  	&Spikes through lag p\\
	&Coefficients may oscillate&$\phi_{ss}=0$ for $s\geq p$\\
MA(1): $\beta>0$& $\rho_{s}=0, for \ s\geq 2$&Oscillating decay: $\phi_{11}>0$\\
MA(1): $\beta>0$& $\rho_{s}=0, for \ s\geq 2$&Decay: $\phi_{11}<0$ \\
ARMA(1, 1): $a_{1}>0$& 	&\\
ARMA(1, 1): $a_{1}<0$& 	&\\
ARMA(p, q)& 	&\\
\hline
\end{tabular}
\end{center}
\end{table}


\subsection{Sample Autoorrelations}
Given that a series is stationary, we can use the sample mean, variance and autocorrelations to estimate the parameters of the actual data-generating process.

The estimates of $\mu$, $\sigma^{2}$ and $\rho$:
\begin{eqnarray*}
\overline{y}&=&\frac{\sum_{t=1}^{T}y_{t}}{T}\\
\hat{\sigma^{2}}&=&\frac{\sum_{t=1}^{T}(y_{t}-\overline{y})^{2}}{T}\\
r_{s}&=&\frac{\sum_{t=s+1}^{T}(y_{t}-\overline{y})(y_{t-s}-\overline{y})}{\sum_{t=1}^{T}(y_{t}-\overline{y})^{2}}
\end{eqnarray*}

Box and Pierce used the sample autocorrelations to form the statistic
\begin{eqnarray*}
Q&=&T\sum_{k=1}^{s}r_{k}^{2}
\end{eqnarray*}

If the data are generated from a stationary ARMA process, Q is asymptotically $\chi^{2}(s)$ distribution. 

Ljung and Box test
\begin{eqnarray*}
Q&=&T(T+2)\sum_{k=1}^{s}\frac{r_{k}^{2}}{T-k} \sim \chi^{2}(s)
\end{eqnarray*}

%asymptotic for stationary
\section[Asymptotics for Dynamic Models]{Asymptotics for Dynamic Models}
\subsection{The Simple Autoregressive Model}


\subsection{Martingale Difference Processes}
The adapted sequence $\{\bm{s}_{t},\mathcal{F}_{t}\}$ is called a martingale if for every $t$ the following conditions hold:
\begin{enumerate}
\item $E|\bm{s}_{t}|<\infty$
\item $E(\bm{s}_{t}\mid\mathcal{F}_{t-1})=\bm{s}_{t-1}$ a.s.
\end{enumerate}

The adapted sequence $\{\bm{x}_{t},\mathcal{F}_{t}\}$ is called a martingale difference (m.d.) if for every $t$ the following conditions hold:
\begin{enumerate}
\item $E|\bm{x}_{t}|<\infty$
\item $E(\bm{s}_{t}\mid\mathcal{F}_{t-1})=0$ a.s.
\end{enumerate}

Here is a fundamental property of m.d. processes.
\begin{theorem}
Let $\{x_{t}\}$ be an m.d. sequence and let $g_{t-1}=g(x_{t-1},x_{t-2},\cdots,)$ be any measurable, integrable function of the lagged values of the sequence. Then $x_{t}g_{t-1}$ is also an m.d., and $x_{t}$ and $g_{t-1}$ are uncorrelated.
\end{theorem}
\begin{proof}
By law of iterated expectations, we have
\begin{eqnarray*}
E(x_{t}g_{t-1})&=&E(E(x_{t}g_{t-1}\mid\mathcal{F}_{t-1}))\\
			&=&E(g_{t-1}E(x_{t}\mid\mathcal{F}_{t-1}))\\
			&=&E(g_{t-1}\cdot 0)\\
			&=&0
\end{eqnarray*}
It means that $x_{t}g_{t-1}$ is also an m.d. For uncorrelation, we show that
\begin{eqnarray*}
Cov(x_{t},g_{t-1})&=&E(x_{t}g_{t-1})-E(x_{t})E(g_{t-1})\\
			   &=&E(x_{t}g_{t-1})-E(E(x_{t}\mid\mathcal{F}_{t-1})E(g_{t-1})\\
			  &=&0-0\\
			&=&0
\end{eqnarray*}
\end{proof}

In particular, putting $g_{t-1}=x_{t-j}$ for any $j>0$, the theorem implies 
\begin{eqnarray*}
Cov(x_{t},x_{t-j})&=&0
\end{eqnarray*}
\textbf{The m.d. property implies uncorrelatedness of the sequence, although it is a weaker property than independence.} 

Given an arbitrary sequence $\{y_{t}\}$ satisfying $E|y_{t}|<\infty$ and $\sigma\left(y_{t},y_{t-1},\cdots\right)\subset \mathcal{F}_{t}$, an m.d. sequence can always be generated as the centred sequence $\{x_{t}\}$ where 
\begin{eqnarray*}
x_{t}&=&y_{t}-E(y_{t}\mid\mathcal{F}_{t-1})
\end{eqnarray*}

We consider the law of large number and central limit theorem for m.d. processes.
\begin{theorem}
Let $\{x_{t}\}$ be an m.d. sequence. Then $\plim \bar{x}_{n}=0$ if either of the following condition hold.
\begin{enumerate}
\item The sequence is strictly stationary and $E|x_{t}|<\infty$
\item $E|x_{t}|^{1+\delta}<\infty$, $\forall t$
\end{enumerate}
\end{theorem}

The following is central limit theorem for m.d.
\begin{theorem}
Let $\{x_{t},\mathcal{F}_{t}\}$ be a m.d. sequence with $E(x_{t}^{2})=\sigma_{t}^{2}$ and let $\bar{\sigma}_{n}^{2}=n^{-1}\sum_{t=1}^{n}\sigma_{t}^{2}$. If
\begin{enumerate}
\item  $\plim n^{-1}\sum_{t=1}^{n}\left(x_{t}^{2}-\sigma_{t}^{2}\right)=0$
\item either 
	\begin{enumerate}
	\item the sequence is strictly stationary
	\item 
	\end{enumerate}
\end{enumerate}
\end{theorem}




%unit root
\section[Trends and Unit Roots]{Trends and Unit Roots}
\subsection{The Random Walk Model}
Consider the following special case AR(1) process:
\begin{eqnarray*}
y_{t}&=&y_{t-1}+\varepsilon_{t}\\
\Delta y_{t}&=&\varepsilon_{t}
\end{eqnarray*}

If $y_{0}$ is a given initial condition, its solution is
\begin{eqnarray*}
y_{t}&=&y_{0}+\sum_{i=1}^{t}\varepsilon_{i}
\end{eqnarray*}

Take expectation
\begin{eqnarray*}
Ey_{t}&=& E\left(y_{0}+\sum_{i=1}^{t}\varepsilon_{i}\right)=y_{0}
\end{eqnarray*}
Taking expectation and variance
\begin{eqnarray*}
E_{t}y_{t+1}&=&E_{t}\left(y_{t}+\varepsilon_{t+1}\right)=y_{t}\\
E_{t}y_{t+s}&=&E_{t}\left(y_{t}+\sum_{i=1}^{s}\varepsilon_{t+i}\right)=y_{t}\\
Var(y_{t})&=&Var\left(\sum_{i=1}^{t}\varepsilon_{i}\right)=t\sigma^{2}\\
Var(y_{t-s})&=&Var\left(\sum_{i=1}^{t-s}\varepsilon_{i}\right)=(t-s)\sigma^{2}\\
E\left[(y_{t}-y_{0})(y_{t-s}-y_{0})\right]&=&E\left[(\sum_{i=1}^{t}\varepsilon_{i} )(\sum_{i=1}^{t+s}\varepsilon_{i})\right]\\
								&=&E\left[(\sum_{i=1}^{t-s}\varepsilon_{i}^{2} )\right]\\
								&=&(t-s)\sigma^{2}
\end{eqnarray*}

The correlation coefficient $\rho_{s}$ is
\begin{eqnarray*}
\rho_{s}&=&\frac{(t-s)\sigma^{2}}{\sqrt{(t-s)\sigma^{2}}\times \sqrt{t\sigma^{2}}}\\
&=&\sqrt{\frac{t-s}{t}}
\end{eqnarray*}

When t is big relative to s, the $\rho_{s}$  are close to unity and decay very slowly. 

\paragraph{The Random Walk plus Drift Model}
Adding a constant term $a_{0}$:
\begin{eqnarray*}
y_{t}&=&y_{t-1}+a_{0}+\varepsilon_{t}
\end{eqnarray*}

Giving the initial condition $y_{0}$, its solution is
\begin{eqnarray*}
y_{t}&=&y_{0}+a_{0}t+\sum_{i=1}^{t}\varepsilon_{i}
\end{eqnarray*}

The behavior of $y_{t}$ is governed by two nonstationary components: a linear deterministic trend and the stochastic trend. 

\subsection{Function Spaces}
\begin{itemize}
\item $x$: an element of $C$, that is, any continuous curve traversing the unit interval, be denoted.
\item Coordinates of $x$: $x(r)\in\mathbb{R}$ is the unique values of $x$ at points $r\in[0,1]$ are called the coordinates of $x$.
\end{itemize}

For two members of $C$, $x\in C$ and $y\in C$, we need to say how close together they are. Technically, $C$ must be assigned a metric. For example, we can define \textbf{Euclidean metric} for any pair of real numbers $x$ and $y$ as $d_{E}(x,y)=|x-y|$. The pair $(\mathbb{R},d_{E})$ is known as the \textbf{Euclidean space}. 
 
We also can define a metric called \textbf{uniform metric} as 
\begin{eqnarray*}
d_{U}(x,y)&=&\sup_{0\leq r\leq 1}|x(r)-y(r)|
\end{eqnarray*}
This is just the \textit{largest vertical separation} between the pair of functions over the interval. $(C,d_{U})$ is a metric space. 

\subsection{Brownian Motion}
A Brownian motion $B$ is a real random function on the unit interval, with the following properties:
\begin{enumerate}
\item $B\in C$ with probability 1.
\item $B(0)=0$ with probability 1.
\item for any set of subintervals defined by arbitrary $0\leq r_{1}<r_{2}<\dots<r_{k}\leq 1$, the increments $B(r_{1}$, $B(r_{2})-B(r_{1}$, $\cdots$, $B(r_{k})-B(r_{k-1})$ are independent.
\item $B(t)-B(s)\sim N(0,t-s)$ for $0\leq s<t\leq 1$.
\end{enumerate}

\subsection{The Functional Central Limit Theorem}
We construct a variable $X_{T}(r)$ from the sample mean of the first $r$th fraction of the observations, $r\in[0,1]$, defined by
\begin{eqnarray*}
X_{T}(r)&\equiv&\frac{1}{T}\sum_{t=1}^{[Tr]}u_{t}
\end{eqnarray*}




\subsection{Dickey-Fuller Tests}
Subtracting $y_{t-1}$ from each side of the equation $y_{t}=a_{1}y_{t-1}+\varepsilon_{t}$, we get $\Delta y_{t}=\gamma y_{t-1}+\varepsilon_{t}$, where $\gamma=a_{1}-1$. Testing the hypothesis $a_{1}=1$ is equivalent to testing $\gamma=0$.

Dickey and Fuller consider three different regression equations 
\begin{eqnarray*}
&&\mbox{random walk model}\\
\Delta y_{t}&=&\gamma y_{t-1}+\varepsilon_{t} \\
&&  \mbox{add a drift}\\
\Delta y_{t}&=&a_{0}+\gamma y_{t-1}+\varepsilon_{t}\\
&&\mbox{add a drift and linear time trend}\\
\Delta y_{t}&=&a_{0}+\gamma y_{t-1}+a_{2}t+\varepsilon_{t} 
\end{eqnarray*}

Run the OLS and get the estimated value of $\gamma$ and associated standard error of these three models. However, the critical values of the t-statistics do depend on whether a drift and/or time trend is included in \textbf{regression models}. Note that the appropriate critical values depend on \textbf{sample size}. For any given level of significance, the critical values of the t-statistic decrease as sample size increases. 

\subsection{Augmented Dicker-Fuller test}
Consider the pth-order autoregressive process:
\begin{eqnarray*}
y_{t}&=&a_{0}+a_{1}y_{t-1}+a_{2}y_{t-2}+a_{3}y_{t-3}+\cdots+a_{p-2}y_{t-p+2}+a_{p-1}y_{t-p+1}+a_{p}y_{t-p}+\varepsilon_{t}
\end{eqnarray*}
jj
Add and subtract $a_{p}y_{t-p+1}$
\begin{eqnarray*}
y_{t}&=&a_{0}+a_{1}y_{t-1}+a_{2}y_{t-2}+\cdots+a_{p-2}y_{t-p+2}+a_{p-1}y_{t-p+1}+a_{p}y_{t-p+1}+a_{p}y_{t-p}-a_{p}y_{t-p+1}+\varepsilon_{t}\\
	&=&a_{0}+a_{1}y_{t-1}+a_{2}y_{t-2}+\cdots+a_{p-2}y_{t-p+2}+(a_{p-1}+a_{p})y_{t-p+1}-a_{p}\Delta y_{t-p+1}+\varepsilon_{t}
\end{eqnarray*}

Add and subtract $(a_{p-1}+a_{p})y_{t-p+2}$
\begin{eqnarray*}
y_{t}&=&a_{0}+\cdots+a_{p-2}y_{t-p+2}+(a_{p-1}+a_{p})y_{t-p+2}+(a_{p-1}+a_{p})y_{t-p+1}-(a_{p-1}+a_{p})y_{t-p+2}-a_{p}\Delta y_{t-p+1}+\varepsilon_{t}\\
	&=&a_{0}+\cdots+(a_{p-2}+a_{p-1}+a_{p})y_{t-p+2}-(a_{p-1}+a_{p})\Delta y_{t-p+2}-a_{p}\Delta y_{t-p+1}+\varepsilon_{t}
\end{eqnarray*}

Continuing in this fashion, we get
\begin{eqnarray*}
\Delta y_{t}&=&a_{0}+\gamma y_{t-1}+\sum_{i=1}^{p}\beta_{i}\Delta y_{t-i+1}+\varepsilon_{t}\\
where\ \ \gamma&=& -\left( 1-\sum_{i=1}^{p}a_{i}\right)\\
	\beta_{i}&=& \sum_{j=i}^{p}a_{j}
\end{eqnarray*}

We can use the same Dickey-Fuller statistics which depends on the regression models and sample size.


%var
\section{Multivarite Model}
\subsection{Vector Autoregression Model}
Consider the first order VAR model (\textbf{Structural equation}):
\begin{eqnarray*}
  y_t & = & b_{10} - b_{12} z_t + \gamma_{11} y_{t - 1} + \gamma_{12} z_{t -
  1} + \varepsilon_{\tmop{yt}}\\
  z_t & = & b_{20} - b_{21} y_t + \gamma_{21} y_{t - 1} + \gamma_{22} z_{t -
  1} + \varepsilon_{\tmop{zt}}\\
&assume&\\
&&\{y_t \}, \ \{z_t \} \ are \ stationary\\
&&\varepsilon_{\tmop{yt}} \sim W.N (0, \sigma_y^2)\\
&&\varepsilon_{\tmop{zt}} \sim W.N (0, \sigma_z^2)\\
&&\varepsilon_{\tmop{yt}}, \ \varepsilon_{\tmop{zt}}\  are \ uncorrelated
\end{eqnarray*}

\begin{eqnarray*}
  y_t + b_{12} z_t & = & b_{10} + \gamma_{11} y_{t - 1} + \gamma_{12} z_{t -
  1} + \varepsilon_{\tmop{yt}}\\
  b_{21} y_t + z_t & = & b_{20} + \gamma_{21} y_{t - 1} + \gamma_{22} z_{t -
  1} + \varepsilon_{\tmop{zt}}\\
  \left(\begin{array}{cc}
    1 & b_{12}\\
    b_{21} & 1
  \end{array}\right) \left(\begin{array}{c}
    y_t\\
    z_t
  \end{array}\right) & = & \left(\begin{array}{c}
    b_{10}\\
    b_{20}
  \end{array}\right) + \left(\begin{array}{cc}
    \gamma_{11} & \gamma_{12}\\
    \gamma_{21} & \gamma_{22}
  \end{array}\right) \left(\begin{array}{c}
    y_{t - 1}\\
    z_{t - 1}
  \end{array}\right) + \left(\begin{array}{c}
    \varepsilon_{\tmop{yt}}\\
    \varepsilon_{\tmop{zt}}
  \end{array}\right)\\
  \tmop{Bx}_t & = & \Gamma_0 + \Gamma_1 x_{t - 1} + \varepsilon_t \\
  x_t & = & B^{- 1} \Gamma_0 + B^{- 1} \Gamma_1 x_{t - 1} + B^{- 1}
  \varepsilon_t\\
  & = & A_0 + A_1 x_{t - 1} + e_t\\
  A_0 & = & B^{- 1} \Gamma_0=\left(\begin{array}{cc}
    1 & b_{12}\\
    b_{21} & 1
  \end{array}\right)^{-1}\left(\begin{array}{c}
    b_{10}\\
    b_{20}
  \end{array}\right)\\
  A_1 & = & B^{- 1} \Gamma_1=\left(\begin{array}{cc}
    1 & b_{12}\\
    b_{21} & 1
  \end{array}\right)^{-1}\left(\begin{array}{cc}
    \gamma_{11} & \gamma_{12}\\
    \gamma_{21} & \gamma_{22}
  \end{array}\right)\\
  e_t & = & B^{- 1} \varepsilon_t=\left(\begin{array}{cc}
    1 & b_{12}\\
    b_{21} & 1
  \end{array}\right)^{-1}\left(\begin{array}{c}
    \varepsilon_{\tmop{yt}}\\
    \varepsilon_{\tmop{zt}}
  \end{array}\right)
\end{eqnarray*}


Rewrite the vector form $x_t = A_0 + A_1 x_{t - 1} + e_t$
\begin{eqnarray*}
  y_t & = & a_{10} + a_{11} y_{t - 1} + a_{12} z_{t - 1} + e_{1 t}\\
  z_t & = & a_{20} + a_{21} y_{t - 1} + a_{22} z_{t - 1} + e_{2 t}
\end{eqnarray*}


We know that
\begin{eqnarray*}
  e_t & = & B^{- 1} \varepsilon_t\\
  & = & \frac{1}{1 - b_{12} b_{21}} \left(\begin{array}{cc}
    1 & - b_{12}\\
    - b_{21} & 1
  \end{array}\right) \left(\begin{array}{c}
    \varepsilon_{\tmop{yt}}\\
    \varepsilon_{\tmop{zt}}
  \end{array}\right)\\
  & = & \frac{1}{1 - b_{12} b_{21}} \left(\begin{array}{c}
    \varepsilon_{\tmop{yt}} - b_{12} \varepsilon_{\tmop{zt}}\\
    \varepsilon_{\tmop{zt}} - b_{21} \varepsilon_{\tmop{yt}}
  \end{array}\right)
\end{eqnarray*}


So
\begin{eqnarray*}
  e_{1 t} & = & \frac{\varepsilon_{\tmop{yt}} - b_{12}
  \varepsilon_{\tmop{zt}}}{1 - b_{12} b_{21}}\\
  &  & \\
  e_{2 t} & = & \frac{\varepsilon_{\tmop{zt}} - b_{21}
  \varepsilon_{\tmop{yt}}}{1 - b_{12} b_{21}}
\end{eqnarray*}


To derive the properties of $e_t$
\begin{eqnarray*}
  E (e_{1 t}) & = & E \left( \frac{\varepsilon_{\tmop{yt}} - b_{12}
  \varepsilon_{\tmop{zt}}}{1 - b_{12} b_{21}} \right) = 0\\
  E (e_{1 t}^2) & = & E \left[ \left( \frac{\varepsilon_{\tmop{yt}} - b_{12}
  \varepsilon_{\tmop{zt}}}{1 - b_{12} b_{21}} \right)^2 \right]\\
  & = & \frac{\sigma^2_{\tmop{yt}} + b^2_{12} \sigma^2_{\tmop{zt}}}{\left( 1
  - b_{12} b_{21} \right)^2}\\
  E \left( e_{1 t} e_{1 t - i} \right) & = & E \left[ \frac{\left(
  \varepsilon_{\tmop{yt}} - b_{12} \varepsilon_{\tmop{zt}} \right) \left(
  \varepsilon_{\tmop{yt} - i} - b_{12} \varepsilon_{\tmop{zt} - i}
  \right)}{\left. ( 1 - b_{12} b_{21} \right)^2} \right] = 0\\
  E \left( e_{1 t} e_{2 t} \right) & = & E \left[ \frac{\left(
  \varepsilon_{\tmop{yt}} - b_{12} \varepsilon_{\tmop{zt}} \right) \left(
  \varepsilon_{\tmop{zt}} - b_{21} \varepsilon_{\tmop{yt}} \right)}{\left. ( 1
  - b_{12} b_{21} \right)^2} \right]\\
  & = & \frac{- \left( b_{21} \sigma^2_{\tmop{yt}} + b_{12}
  \sigma^2_{\tmop{zt}} \right)}{\left. ( 1 - b_{12} b_{21} \right)^2}
\end{eqnarray*}

\subsubsection{Identification}
Compare the number of parameters in the structural VAR with the number of parameters from the standard form VAR model. 
\begin{center}
\begin{tabular}{llll}
\hline
Form&Estimate&Calculation&N\\\hline
Reduced form&$a_{10}$, $a_{20}$, $a_{11}$, $a_{12}$, $a_{21}$, $a_{22}$&$var(e_{1t})$, $var(e_{2t})$, $cov(e_{1t}, e_{2t})$&9\\
Structural form&$b_{10}$, $b_{20}$, $\gamma_{11}$, $\gamma_{12}$, $\gamma_{21}$, $\gamma_{22}$, $b_{12}$, $b_{21}$&$\sigma_y$, $\sigma_z$&10\\
\hline
\end{tabular}
\end{center}

The primitive system contains 10 parameters, whereas the VAR estimation yields only nine parameters. Unless restrict one of the parameters, it is not possible to identify the primitive system. If exactly one parameter of the primitive system is restricted, the system is exactly identified, and if more than one parameter is restricted, the system is over-identified. 


\subsection{VAR}
\begin{align*}
Y_t=A Y_{t-1}+\epsilon_t= A^2 Y_{t-2} + A \epsilon_{t-1} + \epsilon_t = A^m Y_{t-m}+A^{m-1} \epsilon_{t-m+1}+..+\epsilon_t
\end{align*}
This explodes if $A^m$ explodes if $m \rightarrow \infty$, this does not explode if it converges to 0. Thus if all eigenvalues of A are smaller than 1 in absolute value.

\subsection{Vector Auto Regression: VAR}
Definition VAR: for k-variate time series:
\begin{align*}
Y_t=c+ \Gamma_1 Y_{t-1} + ... + \Gamma_p Y_{t-p} + \epsilon_t
\\
Y_t: \text{K-dim} \hspace{10mm} c: \text{K-dim} \hspace{10mm} \Gamma_1: \text{K x H -dim} \hspace{10mm} \epsilon_t: \text{K-dim}
\end{align*}
Probalistic analysis VAR(1) identical to AR(1)
Model:
\begin{align*}
Y_t=A Y_{t-1}+\epsilon_t \text{, } \hspace{10mm} \epsilon_t  \text{ iid}
\end{align*}

Stationary solution: write as VMA($\infty$):
\begin{align*}
Y_t = A^m Y_{t-m}+ \sum_{j=0}^{m-1} A^{j} \epsilon_{t-j} \overset{m \to \infty}{\to} \sum_{j=0}^{\infty} A^{j} \epsilon_{t-j} \hspace{10mm} \text{if all absolute eigenvalues of A $<$ 1.}
\end{align*}
Expectation: 
\begin{align*}
E(Y_t) = E(\sum_{j=0}^{\infty} A^{j} \epsilon_{t-j}) = 0 \text{ if } E(\epsilon_t)=0
\end{align*}
Variance:
\begin{align*}
Var(Y_t)=Var(\sum_{j=0}^{\infty} A^{j} \epsilon_{t-j})=\sum_{j=0}^{\infty} Var(A^{j}
\epsilon_{t-j})=\sum_{j=0}^{\infty} A^{j} Var(\epsilon_{t-j}) (A^{j})^T 
\end{align*}
Alternative:
\begin{align*}
Var(Y_t)= Var(A Y_{t-1} + \epsilon_t) = Var(A Y_{t-1}) + Var(\epsilon_t) = A Var(Y_{t-1}) A^T + Var(\epsilon_t) = A Var(Y_t) A^T + Var(\epsilon_t).
\end{align*}
Last equality due to stationarity. \\
Slides of Sender, VAR(1):
\begin{align*}
Var(Y_{t+k})=\sum_{i=1}^{k} \theta(i)\cdot\Psi\cdot\theta^T(i) \\
\text{with} \hspace{20mm} &\Psi=E(\epsilon_t\cdot\epsilon_t^T) \\
\text{and} \hspace{20mm} &\theta(k)=(I+\phi_1+\phi_1^2+\dots+\phi_1^{k-1})
\end{align*}

\subsection{Impulse response functions}
For instance: What is the effect on the, say, third component of $Y_t$ if $\epsilon_{t-7}$ has a unit shock in its second component $\rightarrow$ look at $(A^7)_{3,2}$.
A not known, but has to be estimated so use Confidence Intervals. To calculate these, use the delta-method.

\begin{comment}
\end{comment}



\section[Trends and Unit Roots]{Trends and Unit Roots}
\subsection{The Random Walk Model}
Consider the following special case AR(1) process:
\begin{eqnarray*}
y_{t}&=&y_{t-1}+u_{t}\\
\Delta y_{t}&=&u_{t}
\end{eqnarray*}

If $y_{0}$ is a given initial condition, its solution is
\begin{eqnarray*}
y_{t}&=&y_{0}+\sum_{i=1}^{t}u_{i}
\end{eqnarray*}

Take expectation
\begin{eqnarray*}
Ey_{t}&=& E\left(y_{0}+\sum_{i=1}^{t}u_{i}\right)=y_{0}
\end{eqnarray*}
Taking expectation and variance
\begin{eqnarray*}
E_{t}y_{t+1}&=&E_{t}\left(y_{t}+\varepsilon_{t+1}\right)=y_{t}\\
E_{t}y_{t+s}&=&E_{t}\left(y_{t}+\sum_{i=1}^{s}\varepsilon_{t+i}\right)=y_{t}\\
Var(y_{t})&=&Var\left(\sum_{i=1}^{t}u_{i}\right)=t\sigma^{2}\\
Var(y_{t-s})&=&Var\left(\sum_{i=1}^{t-s}u_{i}\right)=(t-s)\sigma^{2}\\
E\left[(y_{t}-y_{0})(y_{t-s}-y_{0})\right]&=&E\left[(\sum_{i=1}^{t}u_{i} )(\sum_{i=1}^{t+s}u_{i})\right]\\
								&=&E\left[(\sum_{i=1}^{t-s}u_{i}^{2} )\right]\\
								&=&(t-s)\sigma^{2}
\end{eqnarray*}

The correlation coefficient $\rho_{s}$ is
\begin{eqnarray*}
\rho_{s}&=&\frac{(t-s)\sigma^{2}}{\sqrt{(t-s)\sigma^{2}}\times \sqrt{t\sigma^{2}}}\\
&=&\sqrt{\frac{t-s}{t}}
\end{eqnarray*}

When $t$ is big relative to s, the $\rho_{s}$  are close to unity and decay very slowly. 

\paragraph{The Random Walk plus Drift Model}
Adding a constant term $a_{0}$:
\begin{eqnarray*}
y_{t}&=&y_{t-1}+a_{0}+u_{t}
\end{eqnarray*}

Giving the initial condition $y_{0}$, its solution is
\begin{eqnarray*}
y_{t}&=&y_{0}+a_{0}t+\sum_{i=1}^{t}u_{i}
\end{eqnarray*}

The behavior of $y_{t}$ is governed by two nonstationary components: a linear deterministic trend and the stochastic trend. 

\subsection{Function Spaces}
\begin{itemize}
\item $x$: an element of $C$, that is, any continuous curve traversing the unit interval, be denoted.
\item Coordinates of $x$: $x(r)\in\mathbb{R}$ is the unique values of $x$ at points $r\in[0,1]$ are called the coordinates of $x$.
\end{itemize}

For two members of $C$, $x\in C$ and $y\in C$, we need to say how close together they are. Technically, $C$ must be assigned a metric. For example, we can define \textbf{Euclidean metric} for any pair of real numbers $x$ and $y$ as $d_{E}(x,y)=|x-y|$. The pair $(\mathbb{R},d_{E})$ is known as the \textbf{Euclidean space}. 
 
We also can define a metric called \textbf{uniform metric} as 
\begin{eqnarray*}
d_{U}(x,y)&=&\sup_{0\leq r\leq 1}|x(r)-y(r)|
\end{eqnarray*}
This is just the \textit{largest vertical separation} between the pair of functions over the interval. $(C,d_{U})$ is a metric space. 

\subsection{Brownian Motion}
A Brownian motion $B$ is a real random function on the unit interval, with the following properties:
\begin{enumerate}
\item $B\in C$ with probability 1.
\item $B(0)=0$ with probability 1.
\item for any set of subintervals defined by arbitrary $0\leq r_{1}<r_{2}<\dots<r_{k}\leq 1$, the increments $B(r_{1}$, $B(r_{2})-B(r_{1}$, $\cdots$, $B(r_{k})-B(r_{k-1})$ are independent.
\item $B(t)-B(s)\sim N(0,t-s)$ for $0\leq s<t\leq 1$.
\end{enumerate}

\subsection{The Functional Central Limit Theorem}
We construct a variable $X_{T}(r)$ from the sample mean of the first $r$th fraction of the observations, $r\in[0,1]$, defined by
\begin{eqnarray*}
X_{T}(r)&\equiv&\frac{1}{T}\sum_{t=1}^{[Tr]}u_{t}
\end{eqnarray*}
For any given realization, $X_{T}(r)$ is a step function in $r$, with
\begin{eqnarray*}
X_{T}(r)&=&\begin{cases}
0						&	0\leq r<1/T\\
\frac{u_{1}}{T}				&	1/T\leq r<2/T\\
\frac{u_{1}+u_{2}}{T}		&	2/T\leq r<3/T\\
\vdots					&				\\
\frac{u_{1}+u_{2}+\cdots+u_{T-1}}{T}		&	(T-1)/T\leq r<1\\
\frac{u_{1}+u_{2}+\cdots+u_{T-1}+u_{T}}{T}		&	r=1\\
\end{cases}
\end{eqnarray*}
Then 
\begin{eqnarray*}
\sqrt{T}X_{T}(r)&=&\sqrt{T}\frac{1}{T}\sum_{t=1}^{[Tr]}u_{t}\\
			&=&\frac{1}{\sqrt{T}}\sum_{t=1}^{[Tr]}u_{t}\\
			&=&\frac{\sqrt{[Tr]}}{\sqrt{T}}\frac{1}{\sqrt{[Tr]}}\sum_{t=1}^{[Tr]}u_{t}\\
			&\approx_{d}&\sqrt{r} N(0,\sigma^{2})\\
			&\sim&N(0,r\sigma^{2})
\end{eqnarray*}
The last second comes from the facts that $\frac{\sqrt{[Tr]}}{\sqrt{T}}\to \sqrt{r}$ and $\frac{1}{\sqrt{[Tr]}}\sum_{t=1}^{[Tr]}u_{t}\approx_{d}N(0,\sigma^{2})$.

If we consider a sample mean based on observations $[Tr_{1}]$ through $[Tr_{2}]$ for $r_{2}>r_{1}$, we would conclude that 
\begin{eqnarray*}
\sqrt{T}\left[X_{T}(r_{2})-X_{T}(r_{2})\right]/\sigma&\approx&N(0,r_{2}-r_{1})
\end{eqnarray*}

It should not be surprising that the sequence of stochastic functions $\left\{\sqrt{T}X_{T}(\cdot)/\sigma\right\}_{T=1}^{\infty}$ has an asymptotic probability law that is described by standard Brownian motion
\begin{eqnarray*}
\sqrt{T}X_{T}(\cdot)/\sigma&\to&W(\cdot)
\end{eqnarray*}
This result is called \textit{functional central limit theorem}.
\begin{remark}
We should notice the difference between $X_{T}(\cdot)$ and $X_{T}(r)$. $X_{T}(\cdot)$ denotes a random function while $X_{T}(r)$ denotes the value that function assumes at date $r$; thus, $X_{T}(\cdot)$ is a function, while $X_{T}(r)$ is a random variable.
\end{remark}

\paragraph{Convergence in law for random functions}
Let $S(\cdot)$ be a continuous-time stochastic process with $S(r)$ representing its value at time date $r$ for $r\in [0,1]$. Suppose that for any given realization $S(\cdot)$ is a continuous function of $r$ with probability 1. 
\begin{definition}
For $\left\{S_{T}(\cdot)\right\}_{T=1}^{\infty}$ a sequence of such continuous functions, we say that $S_{T}(\cdot)\xrightarrow{L}S(\cdot)$ if all of the following hold:
\begin{enumerate}
\item For any finite collection of $k$ particular dates $0\leq r_{1}<r_{2}<\cdots<r_{k}\leq 1$, the sequence of $k-$dimensional random vectors $\{\bm{y}_{T}\}_{T=1}^{\infty}$ converges in distribution to the vector $\bm{y}$, where 
	\begin{eqnarray*}
	\bm{y}_{T}\equiv\left[\begin{array}{c}
	S_{T}(r_{1})\\
	S_{T}(r_{2})\\
	\vdots\\
	S_{T}(r_{k})\\
	\end{array}\right]
	&&
	\bm{y}\equiv\left[\begin{array}{c}
	S(r_{1})\\
	S(r_{2})\\
	\vdots\\
	S(r_{k})\\
	\end{array}\right]
	\end{eqnarray*}
\item For each $\varepsilon>0$, the probability that $S_{T}(r_{1})$ differs from $S_{T}(r_{2})$ for any dates $r_{1}$ and  $r_{2}$ within $\delta$ of each other goes to zero uniformly in $T$ as $\delta\to 0$.
\item $P\{|S_{T}(0)|>\lambda\}\to 0$ uniformly in $T$ as $\lambda\to\infty$.
\end{enumerate}
\end{definition}
\begin{remark}
This definition applies to sequences of continuous functions, though the function $X_{T}(r)$ is a discontinuous step function. Fortunately, the discontinuities occur at a countable set of points.
\end{remark}
\paragraph{Convergence in probability to sequences of random functions} Let $\left\{S_{T}(\cdot)\right\}_{T=1}^{\infty}$ and $\left\{V_{T}(\cdot)\right\}_{T=1}^{\infty}$ denote sequences of random continuous functions with $S_{T}:r\in [0,1]\to\mathbb{R}^{1}$ and $V_{T}:r\in [0,1]\to\mathbb{R}^{1}$. Let the scalar $Y_{T}$ represent the largest amount by which $S_{T}(r)$ differs from $V_{T}(r)$ for any $r$ defined by
\begin{eqnarray*}
Y_{T}&\equiv&\sup_{0\leq r\leq 1}|S_{T}(r)-V_{T}(r)|
\end{eqnarray*}
\begin{definition}
If the sequence of scalars $\left\{Y_{T}(\cdot)\right\}_{T=1}^{\infty}$ converges in probability to zero, then we say that the sequence of functions $S_{T}(\cdot)$ converges in probability to $V_{T}(\cdot)$. That is, the expression
\begin{eqnarray*}
S_{T}(\cdot)&\to_{p}&V_{T}(\cdot)
\end{eqnarray*}
is interpreted to mean that
\begin{eqnarray*}
\sup_{r\in[0,1]}|S_{T}(r)-V_{T}(r)|&\to_{p}&0
\end{eqnarray*}
\end{definition}


\paragraph{Continuous mapping theorem}
The result is similar to Slutsky's theorem. 
\begin{theorem}
The continuous mapping theorem states that if $S_{T}(\cdot)\xrightarrow{L}S(\cdot)$ and $g(\cdot)$ is a continuous \textbf{functional}, then $g\left(S_{T}(\cdot)\right)\xrightarrow{L}g\left(S(\cdot)\right)$.
\end{theorem}
For example, consider $S_{T}(r)\equiv [\sqrt{T}X_{T}(r)]^{2}$ and $\sqrt{T}X_{T}(r)\xrightarrow{L}\sigma W(\cdot)$. It follows that
\begin{eqnarray*}
S_{T}(\cdot)&\xrightarrow{L}&\sigma^{2}[W(\cdot)]^{2}
\end{eqnarray*}

\subsection{Unit Root Processes}
A random walk $y_{t}=y_{t-1}+u_{t}$ where $\{u_{t}\}$ is an i.i.d. sequence with zero mean and variance $\sigma^{2}$. If $y_{0}=0$, then 
\begin{eqnarray*}
y_{t}=u_{1}+u_{2}+\cdots+u_{t}
\end{eqnarray*}
The stochastic function $X_{T}(r)$ defined as
\begin{eqnarray*}
X_{T}(r)&=&\begin{cases}
0						&	0\leq r<1/T\\
\frac{y_{1}}{T}				&	1/T\leq r<2/T\\
\frac{y_{2}}{T}				&	2/T\leq r<3/T\\
\vdots					&				\\
\frac{y_{T-1}}{T}			&	(T-1)/T\leq r<1\\
\frac{y_{T}}{T}				&	r=1\\
\end{cases}
\end{eqnarray*}
The figure below plots $X_{T}(r)$ as a function of $r$. The area under this step function is the sum of $T$ rectangles. The integral of $X_{T}(r)$ is thus equivalent to 
\begin{eqnarray*}
\int_{0}^{1}X_{T}(r)dr&=&\frac{y_{1}}{T^{2}}+\frac{y_{2}}{T^{2}}+\cdots+\frac{y_{T-1}}{T^{2}}
\end{eqnarray*}
Then we have
\begin{center}
\definecolor{xdxdff}{rgb}{0.490196078431,0.490196078431,1.}
\definecolor{uuuuuu}{rgb}{0.266666666667,0.266666666667,0.266666666667}
\begin{tikzpicture}[line cap=round,line join=round,>=triangle 45,x=1.0cm,y=1.0cm]
\clip(-2.66,-1.02) rectangle (8.94,6);
\draw [->] (0.,0.) -- (0.,5.);
\node at (0,5.3) {$X_{T}(r)$};
\node at (8,-0.3) {$r$};
\draw [->] (0.,0.) -- (8.,0.);

\begin{scriptsize}
\draw [fill=uuuuuu] (0.,0.) ;
\draw [fill=xdxdff] (0.,6.) ;
\draw [fill=xdxdff] (8.,0.) ;
\draw (1,0) rectangle (2,3);
\node at (1,-0.3) {$\frac{1}{T}$};
\node at (1.5,1.3) {$\frac{y_{1}}{T}$};
\draw (2,0) rectangle (3,2.3);
\node at (2,-0.3) {$\frac{2}{T}$};
\node at (2.5,1.3) {$\frac{y_{2}}{T}$};
\draw (3,0) rectangle (4,3.2);
\node at (3,-0.3) {$\frac{3}{T}$};
\node at (3.5,1.3) {$\frac{y_{3}}{T}$};
\draw (4,0) rectangle (5,2.7);
\node at (4,-0.3) {$\frac{4}{T}$};
\draw (5,0) rectangle (6,3.5);
\node at (5,-0.3) {$\cdots$};
\end{scriptsize}
\end{tikzpicture}
\end{center}

\begin{eqnarray*}
\int_{0}^{1}\sqrt{T}X_{T}(r)dr&=&\sqrt{T}\left[\frac{y_{1}}{T^{2}}+\frac{y_{2}}{T^{2}}+\cdots+\frac{y_{T-1}}{T^{2}}\right]\\
						&=&T^{-3/2}\sum_{t=1}^{T}y_{t-1}
\end{eqnarray*}
By functional central limit theorem and continuous mapping theorem that as $T\to\infty$
\begin{eqnarray*}
\int_{0}^{1}\sqrt{T}X_{T}(r)dr&\xrightarrow{L}&\sigma\int_{0}^{1}W(r)dr
\end{eqnarray*}
which implies that
\begin{eqnarray}
\label{eq6.1}
T^{-3/2}\sum_{t=1}^{T}y_{t-1}&\xrightarrow{L}&\sigma\int_{0}^{1}W(r)dr
\end{eqnarray}

We can write
\begin{eqnarray}
\label{eq6.2}
\nonumber
T^{-3/2}\sum_{t=1}^{T}y_{t-1}&=&T^{-3/2}\left[u_{1}+(u_{1}+u_{2})+\cdots+(u_{1}+u_{2}+\cdots+u_{T-1})\right]\\\nonumber
	&=&T^{-3/2}\left[(T-1)u_{1}+(T-2)u_{2}+\cdots u_{T-1}\right]\\\nonumber
	&=&T^{-3/2}\sum_{t=1}^{T}(T-t)u_{t}\\
	&=&T^{-1/2}\sum_{t=1}^{T}u_{t}-T^{-3/2}\sum_{t=1}^{T}tu_{t}
\end{eqnarray}

From (\ref{eq6.2}), we can derive
\begin{eqnarray}
\label{eq6.3}
\nonumber
T^{-3/2}\sum_{t=1}^{T}tu_{t}&=&T^{-1/2}\sum_{t=1}^{T}u_{t}-T^{-3/2}\sum_{t=1}^{T}y_{t-1}\\
	&\to_{d}&\sigma W(1)-\sigma\int_{0}^{1}W(r)dr
\end{eqnarray}

\begin{theorem}
In summary, for random walk without drift $y_{t}=y_{t-1}+u_{t}$ where $y_{0}=0$ and $\{u_{t}\}$ is an i.i.d. sequence with mean zero and variance $\sigma^{2}$. Then we have

\begin{eqnarray}
\label{eq6.4}
T^{-1/2}\sum_{t=1}^{T}u_{t}&\xrightarrow{L}&\sigma W(1)\\
\label{eq6.5}
T^{-1}\sum_{t=1}^{T}y_{t-1}u_{t}&\xrightarrow{L}&\frac{1}{2}\sigma^{2} [W(1)^{2}-1]\\
\label{eq6.6}
T^{-3/2}\sum_{t=1}^{T}tu_{t}&\xrightarrow{L}&\sigma W(1)-\sigma\int_{0}^{1}W(r)dr\\
\label{eq6.7}
T^{-3/2}\sum_{t=1}^{T}y_{t-1}&\xrightarrow{L}&\sigma\int_{0}^{1}W(r)dr\\
\label{eq6.8}
T^{-2}\sum_{t=1}^{T}y_{t-1}^{2}&\xrightarrow{L}&\sigma^{2}\int_{0}^{1}W(r)^{2}dr\\
\label{eq6.9}
T^{-5/2}\sum_{t=1}^{T}ty_{t-1}&\xrightarrow{L}&\sigma\int_{0}^{1}rW(r)dr\\
\label{eq6.10}
T^{-3}\sum_{t=1}^{T}ty_{t-1}^{2}&\xrightarrow{L}&\sigma^{2}\int_{0}^{1}rW(r)^{2}dr\\
\label{eq6.11}
T^{-(v+1)}\sum_{t=1}^{T}t^{v}&\to&\frac{1}{v+1},\ \ \ v\in \mathbb{N}
\end{eqnarray}
\end{theorem}

\begin{proof}
(\ref{eq6.4}) is trivial. (\ref{eq6.6}) is shown in (\ref{eq6.3}). (\ref{eq6.7}) is shown in (\ref{eq6.1}). Now we prove the remaining properties. 

We define $S_{T}(r)$ as 
\begin{eqnarray*}
S_{T}(r)&=&\left(\sqrt{T}X_{T}(r)\right)^{2}=T[X_{T}(r)]^{2}
\end{eqnarray*}
Then it can be written as a step function 
\begin{eqnarray*}
S_{T}(r)&=&\begin{cases}
0							&	0\leq r<1/T\\
\frac{y_{1}^{2}}{T}				&	1/T\leq r<2/T\\
\frac{y_{2}^{2}}{T}				&	2/T\leq r<3/T\\
\vdots						&				\\
\frac{y_{T-1}^{2}}{T}			&	(T-1)/T\leq r<1\\
\frac{y_{T}^{2}}{T}				&	r=1\\
\end{cases}
\end{eqnarray*}
Then we have
\begin{eqnarray}
\label{eq4.12}
\int_{0}^{1}S_{T}(r)dr&=&\frac{y_{1}^{2}}{T^{2}}+\frac{y_{2}^{2}}{T^{2}}+\cdots+\frac{y_{T-1}^{2}}{T^{2}}=T^{-2}\sum_{t=1}^{T}y_{t-1}^{2}\xrightarrow{L}\sigma^{2}\int_{0}^{1}W(r)^{2}dr
\end{eqnarray}
Now we show (\ref{eq6.8}).

For (\ref{eq6.9}), we prove as
\begin{eqnarray*}
T^{-5/2}\sum_{t=1}^{T}ty_{t-1}&=&T^{-3/2}\sum_{t=1}^{T}\frac{t}{T}y_{t-1}\xrightarrow{L}\sigma\int_{0}^{1}rW(r)dr
\end{eqnarray*}
for $r=t/T$.

For (\ref{eq6.10})
\begin{eqnarray*}
T^{-3}\sum_{t=1}^{T}ty_{t-1}^{2}&=&T^{-2}\sum_{t=1}^{T}(t/T)y_{t-1}^{2}\xrightarrow{L}\sigma^{2}\int_{0}^{1}rW(r)^{2}dr
\end{eqnarray*}
For (\ref{eq6.5})
\begin{eqnarray*}
T^{-1}\sum_{t=1}^{T}y_{t-1}u_{t}&=&\frac{1}{2}\frac{1}{T}y_{T}^{2}-\frac{1}{2}\frac{1}{T}\sum_{t=1}^{T}u_{t}^{2}\\
&=&\frac{1}{2}S_{T}(1)-\frac{1}{2}\frac{1}{T}\sum_{t=1}^{T}u_{t}^{2}\\
&\xrightarrow{L}&\frac{1}{2}\sigma^{2}[W(1)]^{2}-\frac{1}{2}\sigma^{2}\\
&\xrightarrow{L}&\frac{1}{2}\sigma^{2} [W(1)^{2}-1]
\end{eqnarray*}
\end{proof}

\subsubsection{No Constant No Time Trend}
Consider OLS estimation of $\rho$ based on an AR(1) regression
\begin{eqnarray}
\label{eq4.13}
y_{t}&=&\rho y_{t-1}+u_{t}
\end{eqnarray}
where $u_{t}$ is i.i.d. with mean zero and variance $\sigma^{2}$. OLS estimate is
\begin{eqnarray*}
\hat{\rho}_{T}&=&\frac{\sum_{t=1}^{T}y_{t-1}y_{t}}{\sum_{t=1}^{T}y_{t-1}^{2}}
\end{eqnarray*}
when true value of $\rho$ is 1. Then we have
\begin{eqnarray*}
T\left(\hat{\rho}_{T}-1\right)&=&\frac{T^{-1}\sum_{t=1}^{T}y_{t-1}u_{t}}{T^{-2}\sum_{t=1}^{T}y_{t-1}^{2}}
\end{eqnarray*}
For random walk process $y_{t}=y_{0}+u_{1}+u_{2}+\cdots+u_{t}$, apart from the initial term $y_{0}$, by (\ref{eq6.5})
$$T^{-1}\sum_{t=1}^{T}y_{t-1}u_{t}\xrightarrow{L}\frac{1}{2}\sigma^{2} [W(1)^{2}-1]$$
and by (\ref{eq6.8})
$$T^{-2}\sum_{t=1}^{T}y_{t-1}^{2}\xrightarrow{L}\sigma^{2}\int_{0}^{1}W(r)^{2}dr$$
Under the null hypothesis that $\rho=1$, the OLS estimate is characterized by
\begin{eqnarray*}
T\left(\hat{\rho}_{T}-1\right)&=&\frac{T^{-1}\sum_{t=1}^{T}y_{t-1}u_{t}}{T^{-2}\sum_{t=1}^{T}y_{t-1}^{2}}\xrightarrow{L}\frac{\frac{1}{2}[W(1)^{2}-1]}{\int_{0}^{1}W(r)^{2}dr}
\end{eqnarray*}
$[W(1)]^{2}$ is a $\chi^{2}(1)$ variable. 


\subsubsection{Constant No Time Trend}
The DGP is still $y_{t}=y_{t-1}+u_{t}$. Suppose now that a constant term is included in the AR(1) model that is to be estimated by OLS
\begin{eqnarray*}
y_{t}&=&\alpha+\rho y_{t-1}+u_{t}
\end{eqnarray*}


\subsubsection{No Constant No Time Trend}

\subsection{Dickey-Fuller Tests}

\subsubsection{Serial Correlation}
When $\varepsilon_{t}$ is serially correlated, two approaches are developed. 
Subtracting $y_{t-1}$ from each side of the equation $y_{t}=a_{1}y_{t-1}+u_{t}$, we get $\Delta y_{t}=\gamma y_{t-1}+u_{t}$, where $\gamma=a_{1}-1$. Testing the hypothesis $a_{1}=1$ is equivalent to testing $\gamma=0$.

Dickey and Fuller consider three different regression equations 
\begin{eqnarray*}
&&\mbox{random walk model}\\
\Delta y_{t}&=&\gamma y_{t-1}+u_{t} \\
&&  \mbox{add a drift}\\
\Delta y_{t}&=&a_{0}+\gamma y_{t-1}+u_{t}\\
&&\mbox{add a drift and linear time trend}\\
\Delta y_{t}&=&a_{0}+\gamma y_{t-1}+a_{2}t+u_{t} 
\end{eqnarray*}

Run the OLS and get the estimated value of $\gamma$ and associated standard error of these three models. However, the critical values of the t-statistics do depend on whether a drift and/or time trend is included in \textbf{regression models}. Note that the appropriate critical values depend on \textbf{sample size}. For any given level of significance, the critical values of the t-statistic decrease as sample size increases. 

\subsection{Augmented Dicker-Fuller test}
Consider the pth-order autoregressive process:
\begin{eqnarray*}
y_{t}&=&a_{0}+a_{1}y_{t-1}+a_{2}y_{t-2}+a_{3}y_{t-3}+\cdots+a_{p-2}y_{t-p+2}+a_{p-1}y_{t-p+1}+a_{p}y_{t-p}+u_{t}
\end{eqnarray*}
jj
Add and subtract $a_{p}y_{t-p+1}$
\begin{eqnarray*}
y_{t}&=&a_{0}+a_{1}y_{t-1}+a_{2}y_{t-2}+\cdots+a_{p-2}y_{t-p+2}+a_{p-1}y_{t-p+1}+a_{p}y_{t-p+1}+a_{p}y_{t-p}-a_{p}y_{t-p+1}+u_{t}\\
	&=&a_{0}+a_{1}y_{t-1}+a_{2}y_{t-2}+\cdots+a_{p-2}y_{t-p+2}+(a_{p-1}+a_{p})y_{t-p+1}-a_{p}\Delta y_{t-p+1}+u_{t}
\end{eqnarray*}

Add and subtract $(a_{p-1}+a_{p})y_{t-p+2}$
\begin{eqnarray*}
y_{t}&=&a_{0}+\cdots+a_{p-2}y_{t-p+2}+(a_{p-1}+a_{p})y_{t-p+2}+(a_{p-1}+a_{p})y_{t-p+1}-(a_{p-1}+a_{p})y_{t-p+2}-a_{p}\Delta y_{t-p+1}+u_{t}\\
	&=&a_{0}+\cdots+(a_{p-2}+a_{p-1}+a_{p})y_{t-p+2}-(a_{p-1}+a_{p})\Delta y_{t-p+2}-a_{p}\Delta y_{t-p+1}+u_{t}
\end{eqnarray*}

Continuing in this fashion, we get
\begin{eqnarray*}
\Delta y_{t}&=&a_{0}+\gamma y_{t-1}+\sum_{i=1}^{p}\beta_{i}\Delta y_{t-i+1}+u_{t}\\
where\ \ \gamma&=& -\left( 1-\sum_{i=1}^{p}a_{i}\right)\\
	\beta_{i}&=& \sum_{j=i}^{p}a_{j}
\end{eqnarray*}

We can use the same Dickey-Fuller statistics which depends on the regression models and sample size.
\end{document}