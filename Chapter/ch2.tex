\section[Difference Equation]{Difference Equation}
\subsection{The Difference Operator}
First differences
\begin{eqnarray*}
\Delta y_{t}&\equiv&y_{t}-y_{t-1}
\end{eqnarray*}

Second differences
\begin{eqnarray*}
\Delta^{2}y_{t}&\equiv&\Delta(\Delta y_{t})\\
			&\equiv&\Delta y_{t}-\Delta y_{t-1}\\
			&\equiv&(y_{t}-y_{t-1})-(y_{t-1}-y_{t-2})\\
			&\equiv&y_{t}-2y_{t-1}+y_{t-2}\\
\end{eqnarray*}

\subsection{Difference Equations and Solutions}
\begin{itemize}
\item n-th order linear difference equation (with constant coefficient)
\begin{eqnarray*}
y_{t}&=&a_{0}+\sum_{i=1}^{n}a_{i}y_{t-i}+X_{t}
\end{eqnarray*}
where $X_{t}$ is called forcing process. It includes stochastic terms, time trends, and other variables, but not constants and not y, nor lagged values of y.

\item difference form
\begin{eqnarray*}
y_{t}&=&a_{0}+\sum_{i=1}^{n}a_{i}y_{t-i}+X_{t}\\
y_{t}&=&a_{0}+a_{1}y_{t-1}+\sum_{i=2}^{n}a_{i}y_{t-i}+X_{t}\\
y_{t}-y_{t-1}&=&a_{0}+a_{1}y_{t-1}-y_{t-1}+\sum_{i=2}^{n}a_{i}y_{t-i}+X_{t}\\
\Delta y_{t}&=&a_{0}+(a_{1}-1)y_{t-1}+\sum_{i=2}^{n}a_{i}y_{t-i}+X_{t}\\
\end{eqnarray*}

\item A solution of a D.E shows $y_{t}$ equal to a function of the x and t, plus perhaps some initial conditions for y, but not lagged values of y. Solutions are not unique.
\end{itemize}

\subsubsection{Iteration}
Consider following first-order linear difference equation:
\begin{eqnarray*}
y_{t}&=&a_{0}+a_{1}y_{t-1}+\varepsilon_{t}, \ \ y_{0}=y_{0}
\end{eqnarray*}
We know 
\begin{eqnarray*}
y_{1}&=&a_{0}+a_{1}y_{0}+\varepsilon_{1}\\
y_{2}&=&a_{0}+a_{1}y_{1}+\varepsilon_{2}\\
	&=&a_{0}+a_{1}(a_{0}+a_{1}y_{0}+\varepsilon_{1})+\varepsilon_{2}\\
	&=&a_{0}+a_{1}a_{0}+a_{1}^{2}y_{0}+a_{1}\varepsilon_{1}+\varepsilon_{2}\\
y_{3}&=&a_{0}+a_{1}y_{2}+\varepsilon_{3}\\
	&=&a_{0}+a_{1}(a_{0}+a_{1}a_{0}+a_{1}^{2}y_{0}+a_{1}\varepsilon_{1}+\varepsilon_{2})+\varepsilon_{3}\\
	&=&a_{0}+a_{1}a_{0}+a_{1}^{2}a_{0}+a_{1}^{3}y_{0}+a_{1}^{2}\varepsilon_{1}+a_{1}\varepsilon_{2}+\varepsilon_{3}\\
	&\vdots&\\
y_{t}&=&a_{0}(1+a_{1}+a_{1}^{2}+\cdots+a_{1}^{t-1})+a_{1}^{t}y_{0}+a_{1}^{t-1}\varepsilon_{1}+a_{1}^{t-2}\varepsilon_{2}+\cdots+\varepsilon_{t}\\
&=&a_{0}\sum_{i=0}^{t-1}(a_{1})^{i}+a_{1}^{t}y_{0}+\sum_{i=0}^{t-1}(a_{1})^{i}\varepsilon_{t-i}
\end{eqnarray*}

\subsection{Lag Operators}
The lag operator L is defined 
\begin{eqnarray*}
L^{i}y_{t}\equiv y_{t-i}
\end{eqnarray*}
Lag operator has following properties
\begin{enumerate}
\item $L(\beta y_{t})=\beta\cdot Ly_{t}$
\item $L(x_{t}+y_{t})=Lx_{t}+Ly_{t}$
\item $(L^{i}+L^{j})y_{t}=L^{i}y_{t}+L^{j}y_{t}$
\item $L^{i}L^{j}y_{t}=L^{i+j}y_{t}=y_{t-i-j}$
\item $L^{-i}y_{t}=y_{t+i}$
\item For $\left |a\right|<1$
\begin{eqnarray*}
\sum_{i=0}^{+\infty}(a^{i}L^{i})y_{t}&=&\frac{y_{t}}{1-aL}
\end{eqnarray*}
\item For $\left |a\right|>1$
\begin{eqnarray*}
\sum_{i=0}^{+\infty}(a^{-i}L^{-i})y_{t}&=&\frac{-aLy_{t}}{1-aL}
\end{eqnarray*}
\end{enumerate}

\subsubsection{First-Order Difference Equations}
Consider the first-order difference equation below
\begin{eqnarray*}
y_{t}&=&\phi y_{t-1}+w_{t}
\end{eqnarray*}
which can be written using the lag operator as
\begin{eqnarray*}
y_{t}&=&\phi Ly_{t}+w_{t}\\
y_{t}-\phi Ly_{t}&=&w_{t}\\
(1-\phi L)y_{t}&=&w_{t}\\
(1+\phi L+\phi^{2}L^{2}+\cdots+\phi^{t}L^{t})(1-\phi L)y_{t}&=&(1+\phi L+\phi^{2}L^{2}+\cdots+\phi^{t}L^{t})w_{t}\\
\end{eqnarray*}
The compound operator on the left-hand side is
\begin{eqnarray*}
&&(1+\phi L+\phi^{2}L^{2}+\cdots+\phi^{t}L^{t})(1-\phi L)\\
&=&(1+\phi L+\phi^{2}L^{2}+\cdots+\phi^{t}L^{t})\\
&-&(\phi L+\phi^{2}L^{2}+\cdots+\phi^{t}L^{t}+\phi^{t+1}L^{t+1})\\
&=&1-\phi^{t+1}L^{t+1}
\end{eqnarray*}
Then we have
\begin{eqnarray*}
(1-\phi^{t+1}L^{t+1})y_{t}&=&(1+\phi L+\phi^{2}L^{2}+\cdots+\phi^{t}L^{t})w_{t}\\
y_{t}-\phi^{t+1}y_{-1}&=&w_{t}+\phi w_{t-1}+\phi^{2}w_{t-2}+\cdots+\phi^{t}w_{0}\\
y_{t}&=&\phi^{t+1}y_{-1}+w_{t}+\phi w_{t-1}+\phi^{2}w_{t-2}+\cdots+\phi^{t}w_{0}
\end{eqnarray*}

We have known that
\begin{eqnarray*}
(1+\phi L+\phi^{2}L^{2}+\cdots+\phi^{t}L^{t})(1-\phi L)y_{t}&=&y_{t}-\phi^{t+1}y_{-1}
\end{eqnarray*}
if $|\phi|<1$ and $y_{-1}$ is finite, $\phi^{t+1}y_{-1}$ will become negligible as $t$ becomes large
\begin{eqnarray*}
(1+\phi L+\phi^{2}L^{2}+\cdots+\phi^{t}L^{t})(1-\phi L)y_{t}&\cong&y_{t}
\end{eqnarray*}
A sequence $\{y_{t}\}_{t=-\infty}^{\infty}$ is bounded if there exists a finite number $\bar{y}$ such that $|y_{t}|<\bar{y}$ for all $t$. When $|\phi|<1$ and when we are considering applying an operator to a bounded sequence, we have
\begin{eqnarray*}
(1-\phi L)^{-1}&=&\lim_{t\to\infty} (1+\phi L+\phi^{2}L^{2}+\cdots+\phi^{t}L^{t})
\end{eqnarray*}
So we have
\begin{eqnarray*}
(1+\phi L+\phi^{2}L^{2}+\cdots+\phi^{t}L^{t})(1-\phi L)y_{t}&\cong&y_{t}\\
(1-\phi L)^{-1}(1-\phi L)y_{t}&=&y_{t}
\end{eqnarray*}
By this definition, we have
\begin{eqnarray*}
(1-\phi L)y_{t}&=&w_{t}\\
y_{t}&=&(1-\phi L)^{-1}w_{t}\\
&=&w_{t}+\phi w_{-1}+\phi^{2}w_{t-2}+\phi^{3}w_{t-3}+\cdots
\end{eqnarray*}

\subsubsection{Second-Order Difference Equations}
Consider the second-order difference equation
\begin{eqnarray*}
y_{t}&=&\phi_{1}y_{t-1}+\phi_{2}y_{t-2}+w_{t}\\
(1-\phi_{1}L-\phi_{2}L^{2})y_{t}&=&w_{t}
\end{eqnarray*}
The operator in the left-hand side contains a second-order polynomial in the lag operator $L$. Suppose we have
\begin{eqnarray*}
(1-\phi_{1}L-\phi_{2}L^{2})&=&(1-\lambda_{1}L)(1-\lambda_{2}L)\\
&=&(1-[\lambda_{1}+\lambda_{2}]L+\lambda_{1}\lambda_{2}L^{2})
\end{eqnarray*}
Given values for $\phi_{1}$ and $\phi_{2}$, we seek numbers $\lambda_{1}$ and $\lambda_{2}$ such that
\begin{eqnarray*}
\begin{cases}
\lambda_{1}+\lambda_{2}=\phi_{1}\\
\lambda_{1}\lambda_{2}=-\phi_{2}
\end{cases}
\end{eqnarray*}
Two methods to consider this problem.
\begin{itemize}
\item Method 1
\begin{eqnarray*}
(1-\phi_{1}z-\phi_{2}z^{2})&=&(1-\lambda_{1}z)(1-\lambda_{2}z)
\end{eqnarray*}
when $z=\lambda_{1}^{-1}$ or $z=\lambda_{2}^{-1}$, the right hand side is equal to 0. When
\begin{eqnarray*}
\begin{cases}
z_{1}=\frac{\phi_{1}-\sqrt{\phi_{1}^{2}+4\phi_{2}}}{-2\phi_{2}}\\
\\
z_{2}=\frac{\phi_{1}+\sqrt{\phi_{1}^{2}+4\phi_{2}}}{-2\phi_{2}}
\end{cases}
\end{eqnarray*}
the left-hand side is equal to 0 as well. So if we let 
\begin{eqnarray*}
\begin{cases}
\lambda_{1}^{-1}=z_{1}=\frac{\phi_{1}-\sqrt{\phi_{1}^{2}+4\phi_{2}}}{-2\phi_{2}}\\
\\
\lambda_{2}^{-1}=z_{2}=\frac{\phi_{1}+\sqrt{\phi_{1}^{2}+4\phi_{2}}}{-2\phi_{2}}
\end{cases}
\end{eqnarray*}
both sides are equal to 0.
\item Method 2\\
It is easy to find that $\lambda_{1}$ and $\lambda_{2}$ are roots of equation 
\begin{eqnarray*}
\lambda^{2}-\phi_{1}\lambda-\phi_{2}&=&0
\end{eqnarray*}
so 
\begin{eqnarray*}
\begin{cases}
\lambda_{1}^{-1}=\frac{\phi_{1}+\sqrt{\phi_{1}^{2}+4\phi_{2}}}{2}\\
\\
\lambda_{2}^{-1}=\frac{\phi_{1}-\sqrt{\phi_{1}^{2}+4\phi_{2}}}{2}
\end{cases}
\end{eqnarray*}
\end{itemize}

When we have $\lambda_{1}$ and $\lambda_{2}$ by the above two methods, we have 
\begin{eqnarray*}
(1-\lambda_{1}L)(1-\lambda_{2}L)y_{t}&=&w_{t}\\
y_{t}&=&(1-\lambda_{1}L)^{-1}(1-\lambda_{2}L)^{-1}w_{t}
\end{eqnarray*}

If $\lambda_{1}\neq \lambda_{2}$, we define
\begin{eqnarray*}
&&(\lambda_{1}-\lambda_{2})^{-1}\left(\frac{\lambda_{1}}{1-\lambda_{1}L}-\frac{\lambda_{2}}{1-\lambda_{2}L}\right)\\
&=&(\lambda_{1}-\lambda_{2})^{-1}\frac{\lambda_{1}(1-\lambda_{2}L)-\lambda_{2}(1-\lambda_{1}L)}{(1-\lambda_{1}L)(1-\lambda_{2}L)}\\
&=&(\lambda_{1}-\lambda_{2})^{-1}\frac{\lambda_{1}-\lambda_{2}}{(1-\lambda_{1}L)(1-\lambda_{2}L)}\\
&=&\frac{1}{(1-\lambda_{1}L)(1-\lambda_{2}L)}
\end{eqnarray*}
So we have
\begin{eqnarray*}
y_{t}&=&(1-\lambda_{1}L)^{-1}(1-\lambda_{2}L)^{-1}w_{t}\\
&=&(\lambda_{1}-\lambda_{2})^{-1}\left(\frac{\lambda_{1}}{1-\lambda_{1}L}-\frac{\lambda_{2}}{1-\lambda_{2}L}\right)w_{t}\\
&=&\frac{\lambda_{1}}{\lambda_{1}-\lambda_{2}}\frac{1}{1-\lambda_{1}L}w_{t}-\frac{\lambda_{2}}{\lambda_{1}-\lambda_{2}}\frac{1}{1-\lambda_{2}L}w_{t}\\
&=&\frac{\lambda_{1}}{\lambda_{1}-\lambda_{2}}(1+\lambda_{1}L+\lambda_{1}^{2}L^{2}+\lambda_{1}^{3}L^{3}+\cdots)w_{t}\\
&-&\frac{\lambda_{2}}{\lambda_{1}-\lambda_{2}}(1+\lambda_{2}L+\lambda_{2}^{2}L^{2}+\lambda_{2}^{3}L^{3}+\cdots)w_{t}\\
&=&\frac{\lambda_{1}}{\lambda_{1}-\lambda_{2}}(w_{t}+\lambda_{1}w_{t-1}+\lambda_{1}^{2}w_{t-2}+\lambda_{1}^{3}w_{t-3}+\cdots)\\
&-&\frac{\lambda_{2}}{\lambda_{1}-\lambda_{2}}(w_{t}+\lambda_{2}w_{t-1}+\lambda_{2}^{2}w_{t-2}+\lambda_{2}^{3}w_{t-3}+\cdots)\\
&=&(c_{1}+c_{2})w_{t}+(c_{1}\lambda_{1}+c_{2}\lambda_{2})w_{t-1}+(c_{1}\lambda_{1}^{2}+c_{2}\lambda_{2}^{2})w_{t-2}+\cdots
\end{eqnarray*}
where $c_{1}=\lambda_{1}/(\lambda_{1}-\lambda_{2})$ and $c_{2}=-\lambda_{2}/(\lambda_{1}-\lambda_{2})$.

\subsubsection{$p$th-Order Difference Equations}