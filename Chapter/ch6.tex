\section[Trends and Unit Roots]{Trends and Unit Roots}
\subsection{The Random Walk Model}
Consider the following special case AR(1) process:
\begin{eqnarray*}
y_{t}&=&y_{t-1}+\varepsilon_{t}\\
\Delta y_{t}&=&\varepsilon_{t}
\end{eqnarray*}

If $y_{0}$ is a given initial condition, its solution is
\begin{eqnarray*}
y_{t}&=&y_{0}+\sum_{i=1}^{t}\varepsilon_{i}
\end{eqnarray*}

Take expectation
\begin{eqnarray*}
Ey_{t}&=& E\left(y_{0}+\sum_{i=1}^{t}\varepsilon_{i}\right)=y_{0}
\end{eqnarray*}
Taking expectation and variance
\begin{eqnarray*}
E_{t}y_{t+1}&=&E_{t}\left(y_{t}+\varepsilon_{t+1}\right)=y_{t}\\
E_{t}y_{t+s}&=&E_{t}\left(y_{t}+\sum_{i=1}^{s}\varepsilon_{t+i}\right)=y_{t}\\
Var(y_{t})&=&Var\left(\sum_{i=1}^{t}\varepsilon_{i}\right)=t\sigma^{2}\\
Var(y_{t-s})&=&Var\left(\sum_{i=1}^{t-s}\varepsilon_{i}\right)=(t-s)\sigma^{2}\\
E\left[(y_{t}-y_{0})(y_{t-s}-y_{0})\right]&=&E\left[(\sum_{i=1}^{t}\varepsilon_{i} )(\sum_{i=1}^{t+s}\varepsilon_{i})\right]\\
								&=&E\left[(\sum_{i=1}^{t-s}\varepsilon_{i}^{2} )\right]\\
								&=&(t-s)\sigma^{2}
\end{eqnarray*}

The correlation coefficient $\rho_{s}$ is
\begin{eqnarray*}
\rho_{s}&=&\frac{(t-s)\sigma^{2}}{\sqrt{(t-s)\sigma^{2}}\times \sqrt{t\sigma^{2}}}\\
&=&\sqrt{\frac{t-s}{t}}
\end{eqnarray*}

When t is big relative to s, the $\rho_{s}$  are close to unity and decay very slowly. 

\paragraph{The Random Walk plus Drift Model}
Adding a constant term $a_{0}$:
\begin{eqnarray*}
y_{t}&=&y_{t-1}+a_{0}+\varepsilon_{t}
\end{eqnarray*}

Giving the initial condition $y_{0}$, its solution is
\begin{eqnarray*}
y_{t}&=&y_{0}+a_{0}t+\sum_{i=1}^{t}\varepsilon_{i}
\end{eqnarray*}

The behavior of $y_{t}$ is governed by two nonstationary components: a linear deterministic trend and the stochastic trend. 

\subsection{Function Spaces}
\begin{itemize}
\item $x$: an element of $C$, that is, any continuous curve traversing the unit interval, be denoted.
\item Coordinates of $x$: $x(r)\in\mathbb{R}$ is the unique values of $x$ at points $r\in[0,1]$ are called the coordinates of $x$.
\end{itemize}

For two members of $C$, $x\in C$ and $y\in C$, we need to say how close together they are. Technically, $C$ must be assigned a metric. For example, we can define \textbf{Euclidean metric} for any pair of real numbers $x$ and $y$ as $d_{E}(x,y)=|x-y|$. The pair $(\mathbb{R},d_{E})$ is known as the \textbf{Euclidean space}. 
 
We also can define a metric called \textbf{uniform metric} as 
\begin{eqnarray*}
d_{U}(x,y)&=&\sup_{0\leq r\leq 1}|x(r)-y(r)|
\end{eqnarray*}
This is just the \textit{largest vertical separation} between the pair of functions over the interval. $(C,d_{U})$ is a metric space. 

\subsection{Brownian Motion}
A Brownian motion $B$ is a real random function on the unit interval, with the following properties:
\begin{enumerate}
\item $B\in C$ with probability 1.
\item $B(0)=0$ with probability 1.
\item for any set of subintervals defined by arbitrary $0\leq r_{1}<r_{2}<\dots<r_{k}\leq 1$, the increments $B(r_{1}$, $B(r_{2})-B(r_{1}$, $\cdots$, $B(r_{k})-B(r_{k-1})$ are independent.
\item $B(t)-B(s)\sim N(0,t-s)$ for $0\leq s<t\leq 1$.
\end{enumerate}

\subsection{The Functional Central Limit Theorem}
We construct a variable $X_{T}(r)$ from the sample mean of the first $r$th fraction of the observations, $r\in[0,1]$, defined by
\begin{eqnarray*}
X_{T}(r)&\equiv&\frac{1}{T}\sum_{t=1}^{[Tr]}u_{t}
\end{eqnarray*}




\subsection{Dickey-Fuller Tests}
Subtracting $y_{t-1}$ from each side of the equation $y_{t}=a_{1}y_{t-1}+\varepsilon_{t}$, we get $\Delta y_{t}=\gamma y_{t-1}+\varepsilon_{t}$, where $\gamma=a_{1}-1$. Testing the hypothesis $a_{1}=1$ is equivalent to testing $\gamma=0$.

Dickey and Fuller consider three different regression equations 
\begin{eqnarray*}
&&\mbox{random walk model}\\
\Delta y_{t}&=&\gamma y_{t-1}+\varepsilon_{t} \\
&&  \mbox{add a drift}\\
\Delta y_{t}&=&a_{0}+\gamma y_{t-1}+\varepsilon_{t}\\
&&\mbox{add a drift and linear time trend}\\
\Delta y_{t}&=&a_{0}+\gamma y_{t-1}+a_{2}t+\varepsilon_{t} 
\end{eqnarray*}

Run the OLS and get the estimated value of $\gamma$ and associated standard error of these three models. However, the critical values of the t-statistics do depend on whether a drift and/or time trend is included in \textbf{regression models}. Note that the appropriate critical values depend on \textbf{sample size}. For any given level of significance, the critical values of the t-statistic decrease as sample size increases. 

\subsection{Augmented Dicker-Fuller test}
Consider the pth-order autoregressive process:
\begin{eqnarray*}
y_{t}&=&a_{0}+a_{1}y_{t-1}+a_{2}y_{t-2}+a_{3}y_{t-3}+\cdots+a_{p-2}y_{t-p+2}+a_{p-1}y_{t-p+1}+a_{p}y_{t-p}+\varepsilon_{t}
\end{eqnarray*}
jj
Add and subtract $a_{p}y_{t-p+1}$
\begin{eqnarray*}
y_{t}&=&a_{0}+a_{1}y_{t-1}+a_{2}y_{t-2}+\cdots+a_{p-2}y_{t-p+2}+a_{p-1}y_{t-p+1}+a_{p}y_{t-p+1}+a_{p}y_{t-p}-a_{p}y_{t-p+1}+\varepsilon_{t}\\
	&=&a_{0}+a_{1}y_{t-1}+a_{2}y_{t-2}+\cdots+a_{p-2}y_{t-p+2}+(a_{p-1}+a_{p})y_{t-p+1}-a_{p}\Delta y_{t-p+1}+\varepsilon_{t}
\end{eqnarray*}

Add and subtract $(a_{p-1}+a_{p})y_{t-p+2}$
\begin{eqnarray*}
y_{t}&=&a_{0}+\cdots+a_{p-2}y_{t-p+2}+(a_{p-1}+a_{p})y_{t-p+2}+(a_{p-1}+a_{p})y_{t-p+1}-(a_{p-1}+a_{p})y_{t-p+2}-a_{p}\Delta y_{t-p+1}+\varepsilon_{t}\\
	&=&a_{0}+\cdots+(a_{p-2}+a_{p-1}+a_{p})y_{t-p+2}-(a_{p-1}+a_{p})\Delta y_{t-p+2}-a_{p}\Delta y_{t-p+1}+\varepsilon_{t}
\end{eqnarray*}

Continuing in this fashion, we get
\begin{eqnarray*}
\Delta y_{t}&=&a_{0}+\gamma y_{t-1}+\sum_{i=1}^{p}\beta_{i}\Delta y_{t-i+1}+\varepsilon_{t}\\
where\ \ \gamma&=& -\left( 1-\sum_{i=1}^{p}a_{i}\right)\\
	\beta_{i}&=& \sum_{j=i}^{p}a_{j}
\end{eqnarray*}

We can use the same Dickey-Fuller statistics which depends on the regression models and sample size.