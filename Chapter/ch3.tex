\section[ARMA]{ARMA}
The autoregressive moving average (ARMA) model
\begin{eqnarray*}
y_{t}&=&a_{0}+\sum_{i=1}^{p}a_{i}y_{t-i}+\sum_{i=0}^{q}\beta_{i}\varepsilon_{t-i}
\end{eqnarray*}

\subsection{Stationarity}
A stochastic process is \textit{covariance stationary} or \textit{weakly stationary} if for all t and s
\begin{eqnarray*}
E(y_{t})&=&E(y_{t-s})=\mu\\
E\left[(y_{t}-\mu)^{2}\right]&=&E\left[(y_{t-s}-\mu)^{2}\right]=\sigma_{y}^{2}=\gamma_{0}\\
E\left[(y_{t}-\mu)(y_{t-s}-\mu)\right]&=&E\left[(y_{t-j}-\mu)(y_{t-j-s}-\mu)\right]=\gamma_{s}
\end{eqnarray*}
If a process is covariance stationary, the covariance between $y_{t}$ and $y_{t-s}$ depends only on $s$, the length of time separating the observations. It follows that for a covariance stationary process, $\gamma_{s}$ and $\gamma_{-s}$ would represent the same magnitude. 


For a covariance stationary series, we can define the autocorrelation between $y_{t}$ and $y_{t-s}$
\begin{eqnarray*}
\rho_{s}&=&\frac{\gamma_{s}}{\gamma_{0}}
\end{eqnarray*}
where $\gamma_{0}$ is the variance of $y_{t}$

\subsection{Ergodicity}
Imagine a battery of $I$ computers generating sequences $\{y_{t}^{(1)}\}_{t=-\infty}^{\infty}$, $\{y_{t}^{(2)}\}_{t=-\infty}^{\infty}$, $\dots$, $\{y_{t}^{(I)}\}_{t=-\infty}^{\infty}$ and consider selecting the observation associated with date $t$ from each sequence:
$$\{y_{t}^{(1)},y_{t}^{(2)},\dots,y_{t}^{(I)}\}$$
This would be described as a sample of $I$ realizations of the random variable $Y_{t}$. The expectation of the $t$th observation of a time series refers to the mean of the probability distribution
\begin{eqnarray*}
E(Y_{t})&=&\int_{-\infty}^{\infty}y_{t}f_{Y_{t}}(y_{t})dy_{t}
\end{eqnarray*}
We might view this as the probability limit of the ensemble average
\begin{eqnarray*}
E(Y_{t})&=&\plim_{I\to\infty}\frac{1}{I}\sum_{i=1}^{I}Y_{t}^{(i)}
\end{eqnarray*}

The above expectations of a time series in terms of ensemble averages may seem a bit contrived. Usually we have a single realization of size $T$ from the process 
$$\{y_{1}^{(1)},y_{2}^{(1)},\dots,y_{T}^{(1)}\}$$
From these observations we would calculate the sample mean $\bar{y}$, which is a time average
\begin{eqnarray*}
\bar{y}&=&\frac{1}{T}\sum_{t=1}^{T}y_{t}^{(1)}
\end{eqnarray*}

A covariance stationary process is said to be \textit{ergodic for the mean} if $\bar{y}$ converges in probability to $E(Y_{t})$ as $T\to\infty$. 

\subsection{Moving Average Processes}
\subsubsection{The First-Order Moving Average Process}
Let $\{\varepsilon_{t}\}$ be white noise and consider the process
\begin{eqnarray*}
y_{t}&=&\mu+\varepsilon_{t}+\theta\varepsilon_{t-1}
\end{eqnarray*}
where $\mu$ and $\theta$ could be any constants. This time series is called a first-order moving average process, denoted $MA(1)$.

\paragraph{Expectation}
The expectation of $y_{t}$ is 
\begin{eqnarray*}
E(y_{t})&=&E(\mu+\varepsilon_{t}+\theta\varepsilon_{t-1})=\mu+E(\varepsilon)+\theta E(\varepsilon_{t-1})=\mu
\end{eqnarray*}

\paragraph{Variance}
The variance of $y_{t}$ is 
\begin{eqnarray*}
E(y_{t}-\mu)^{2}&=&E(\varepsilon_{t}+\theta\varepsilon_{t-1})^{2}\\
&=&E\left(\varepsilon_{t}^{2}+2\theta\varepsilon_{t}\varepsilon_{t-1}+\theta^{2}\varepsilon_{t-1}^{2}\right)\\
&=&\sigma^{2}+0+\theta^{2}\sigma^{2}\\
&=&(1+\theta^{2})\sigma^{2}
\end{eqnarray*}

\paragraph{Autocovariance}
The first autocovariance of $y_{t}$ is 
\begin{eqnarray*}
E(y_{t}-\mu)(y_{t-1}-\mu)&=&E(\varepsilon_{t}+\theta\varepsilon_{t-1})(\varepsilon_{t-1}+\theta\varepsilon_{t-2})\\
&=&E\left(\varepsilon_{t}\varepsilon_{t-1}+\theta\varepsilon_{t-1}^{2}+\theta\varepsilon_{t}\varepsilon_{t-2}+\theta^{2}\varepsilon_{t-1}\varepsilon_{t-2}\right)\\
&=&0+\theta\sigma^{2}+0+0\\
&=&\theta^{2}\sigma^{2}
\end{eqnarray*}

Higher autocovariances are all zero. For all $j>1$
\begin{eqnarray*}
E(y_{t}-\mu)(y_{t-j}-\mu)&=&0
\end{eqnarray*}

\paragraph{Autocorrelation}
The $j$th autocorrelation of a covariance stationary process is $\rho_{j}=\frac{\gamma_{j}}{\gamma_{0}}$
\begin{eqnarray*}
\rho_{1}&=&\frac{\gamma_{1}}{\gamma_{0}}=\frac{\theta^{2}\sigma^{2}}{(1+\theta^{2})\sigma^{2}}=\frac{\theta^{2}}{1+\theta^{2}}\\
\rho_{j}&=&0,\ \ \text{$j>1$}
\end{eqnarray*}


\subsubsection{The $q$th-Order Moving Average Process}
A $q$th-order moving average process, denoted $MA(q)$, is characterized by
\begin{eqnarray*}
y_{t}&=&\mu+\varepsilon_{t}+\theta_{1}\varepsilon_{t-1}+\theta_{2}\varepsilon_{t-2}+\cdots+\theta_{q}\varepsilon_{t-q}
\end{eqnarray*}

\paragraph{Expectation}
The expectation of $y_{t}$ is 
\begin{eqnarray*}
E(y_{t})&=&E(\mu)+E(\varepsilon_{t})+\theta_{1}E(\varepsilon_{t-1})+\theta_{2}E(\varepsilon_{t-2})+\cdots+\theta_{q}E(\varepsilon_{t-q})=\mu
\end{eqnarray*}

\paragraph{Variance}
The variance of $y_{t}$ is 
\begin{eqnarray*}
E(y_{t}-\mu)^{2}&=&E(\varepsilon_{t}+\theta_{1}\varepsilon_{t-1}+\theta_{2}\varepsilon_{t-2}+\cdots+\theta_{q}\varepsilon_{t-q})^{2}\\
&=&E(\varepsilon_{t})^{2}+E(\theta_{1}\varepsilon_{t-1})^{2}+E(\theta_{2}\varepsilon_{t-2})^{2}+\cdots+E(\theta_{q}\varepsilon_{t-q})^{2}\\
&=&\sigma^{2}+\theta_{1}^{2}\sigma^{2}+\theta_{2}^{2}\sigma^{2}+\cdot+\theta_{q}^{2}\sigma^{2}\\
&=&(1+\theta_{1}^{2}+\theta_{2}^{2}+\cdot+\theta_{q}^{2})\sigma^{2}
\end{eqnarray*}

\paragraph{Autocovariance}
The autocovariance of $y_{t}$ is 
\begin{eqnarray*}
E(y_{t}-\mu)(y_{t-j}-\mu)&=&E(\varepsilon_{t}+\theta_{1}\varepsilon_{t-1}+\theta_{2}\varepsilon_{t-2}+\cdots+\theta_{q}\varepsilon_{t-q})\\
&&(\varepsilon_{t-j}+\theta_{1}\varepsilon_{t-j-1}+\theta_{2}\varepsilon_{t-j-2}+\cdots+\theta_{q}\varepsilon_{t-j-q})\\
&=&E\left(\theta_{j}\varepsilon_{t-j}^{2}+\theta_{j+1}\theta_{1}\varepsilon_{t-j-1}^{2}+\theta_{j+2}\theta_{2}\varepsilon_{t-j-2}^{2}+\cdots+\theta_{q}\theta_{q-j}\varepsilon_{t-q}^{2}\right)\\
&=&\left(\theta_{j}+\theta_{j+1}\theta_{1}+\theta_{j+2}\theta_{2}+\cdots+\theta_{q}\theta_{q-j}\right)\sigma^{2}, \ \ \text{$j=1,2,\dots,q$}
\end{eqnarray*}

For all $j>q$
\begin{eqnarray*}
E(y_{t}-\mu)(y_{t-j}-\mu)&=&0
\end{eqnarray*}

\subsubsection{The Infinite-Order Moving Average Process}
Consider the process when $q\to\infty$ 
\begin{eqnarray*}
y_{t}&=&\mu+\sum_{j=0}^{\infty}\psi_{j}\varepsilon_{t-j}=\mu+\psi_{0}\varepsilon_{t}+\psi_{1}\varepsilon_{t-1}+\psi_{2}\varepsilon_{t-2}+\cdots
\end{eqnarray*}
This could be described as an $MA(\infty)$ process.

The $MA(\infty)$ process is covariance stationary if it is square summable
\begin{eqnarray*}
\sum_{j=0}^{\infty}\psi_{j}^{2}&<&\infty
\end{eqnarray*}
It is often to work with a slightly stronger condition called absolutely summable
\begin{eqnarray*}
\sum_{j=0}^{\infty}|\psi_{j}|&<&\infty
\end{eqnarray*}

\paragraph{Expectation}
The mean of an $MA(\infty)$ process with absolutely summable is
\begin{eqnarray*}
E(y_{t})&=&\lim_{T\infty}E(\mu+\psi_{0}\varepsilon_{t}+\psi_{1}\varepsilon_{t-1}+\psi_{2}\varepsilon_{t-2}+\cdots+\psi_{T}\varepsilon_{t-T})=\mu
\end{eqnarray*}

\paragraph{Autocovariance}
The autocovariance of an $MA(\infty)$ process with absolutely summable is
\begin{eqnarray*}
\gamma_{0}&=&E(y_{t}-\mu)^{2}\\
&=&\lim_{T\infty}E(\psi_{0}\varepsilon_{t}+\psi_{1}\varepsilon_{t-1}+\psi_{2}\varepsilon_{t-2}+\cdots+\psi_{T}\varepsilon_{t-T})^{2}\\
&=&\lim_{T\to\infty}(\psi_{0}^{2}+\psi_{1}^{2}+\psi_{2}^{2}+\cdots+\psi_{T}^{2})\sigma^{2}\\
\gamma_{j}&=&E(y_{t}-\mu)(y_{t-j}-\mu)\\
&=&(\psi_{j}\psi_{0}+\psi_{j+1}\psi_{1}+\psi_{j+2}\psi_{2}+\psi_{j+3}\psi_{3}+\cdots)\sigma^{2}
\end{eqnarray*}

\subsection{Autoregressive Processes}
\subsubsection{The First-Order Autoregressive Process}
A first-order autoregressive, denoted $AR(1)$, satisfies the following difference equation
\begin{eqnarray*}
y_{t}&=&c+\phi y_{t-1}+\varepsilon_{t}
\end{eqnarray*}
When $|\phi|<1$, this process is covariance stationary. The $AR(1)$ process can be rewritten as $MA(\infty)$ process as follows
\begin{eqnarray*}
y_{t}&=&c+\phi y_{t-1}+\varepsilon_{t}\\
&=&c+\varepsilon_{t}+\phi (c+\varepsilon_{t-1})+\phi^{2} y_{t-2}\\
&=&c+\varepsilon_{t}+\phi (c+\varepsilon_{t-1})+\phi^{2} (c+\varepsilon_{t-2})+\phi^{3} y_{t-3}\\
&=&c+\varepsilon_{t}+\phi (c+\varepsilon_{t-1})+\phi^{2} (c+\varepsilon_{t-2})+\phi^{3} (c+\varepsilon_{t-3})+\phi^{4} y_{t-4}\\
&=&c+\varepsilon_{t}+\phi (c+\varepsilon_{t-1})+\phi^{2} (c+\varepsilon_{t-2})+\phi^{3} (c+\varepsilon_{t-3})+\cdots\\
&=&(c+\phi c+\phi^{2}c+\phi^{3}c+\cdots)+\varepsilon_{t}+\phi\varepsilon_{t-1}+\phi^{2}\varepsilon_{t-2}+\phi^{3}\varepsilon_{t-3}+\cdots\\
&=&\frac{c}{1-\phi}+\varepsilon_{t}+\phi\varepsilon_{t-1}+\phi^{2}\varepsilon_{t-2}+\phi^{3}\varepsilon_{t-3}+\cdots
\end{eqnarray*}

We can derive the expectation and autocovariance of $AR(1)$ by the above corresponding $MA(\infty)$ process. We also can derive them by assuming $AR(1)$ process is covariance stationary.

\paragraph{Expectation}
Taking expectations both sides
\begin{eqnarray*}
E(y_{t})&=&c+\phi E(y_{t-1})+E(\varepsilon_{t})\\
\mu&=&c+\phi\mu\\
\mu&=&\frac{c}{1-\phi}
\end{eqnarray*}
\paragraph{Autocovariance}



\subsubsection{The $q$th-Order Autoregressive Process}

\subsection{The Autocorrelation Function}
Autocorrelation function (ACF) and the partial autocorrelation function (PACF) are useful to determine the type of time series data.

For AR(1) model
\begin{itemize}
\item Method 1
\begin{eqnarray*}
y_{t}&=&a_{0}+a_{1}y_{t-1}+\varepsilon_{t}\ \ assume \ stationary\\
E(y_{t})&=&a_{0}+a_{1}E(y_{t-1})+E(\varepsilon_{t})\\
\Rightarrow\mu&=&a_{0}+a_{1}\mu\\
\Rightarrow\mu&=&\frac{a_{0}}{1-a_{1}}\\
Var(y_{t})&=&a_{1}^{2}Var(y_{t-1})+Var(\varepsilon_{t})\\
\Rightarrow \gamma_{0}&=&a_{1}^{2}\gamma_{0}+\sigma^{2}\\
\Rightarrow \gamma_{0}&=&\frac{\sigma^{2}}{1-a_{1}^{2}}\\
Cov(y_{t}, y_{t-s})&=&Cov(a_{0}+a_{1}y_{t-1}+\varepsilon_{t}, a_{0}+a_{1}y_{t-s-1}+\varepsilon_{t-s})\\
			    &=&Cov(a_{1}y_{t-1}+\varepsilon_{t}, a_{1}y_{t-s-1}+\varepsilon_{t-s})\\
\Rightarrow \gamma_{s}&=&a_{1}^{2}\gamma_{s}+a_{1}^{s}\sigma^{2}\\
\Rightarrow \gamma_{s}&=&\frac{a_{1}^{s}\sigma^{2}}{1-a_{1}^{2}}\\
\end{eqnarray*}
So the autocorrelation function for AR(1)
\begin{eqnarray*}
\rho_{s}&=&\frac{\gamma_{s}}{\gamma_{0}}\\
&=&a_{1}^{s}
\end{eqnarray*}

\item Method 2 \\
If the process started at time zero
\begin{eqnarray*}
y_{t}&=&a_{0}\sum_{i=0}^{t-1}a_{1}^{i}+a_{1}^{t}y_{0}+\sum_{i=0}^{t-1}a_{1}^{i}\varepsilon_{t-i}
\end{eqnarray*}
Take the expectation of $y_{t}$ and $y_{t+s}$
\begin{eqnarray*}
E(y_{t})&=&a_{0}\sum_{i=0}^{t-1}a_{1}^{i}+a_{1}^{t}y_{0}\\
E(y_{t+s})&=&a_{0}\sum_{i=0}^{t+s-1}a_{1}^{i}+a_{1}^{t+s}y_{0}
\end{eqnarray*}
If $|a_{1}|<1$, as $t\rightarrow \infty$
\begin{eqnarray*}
\lim_{t\rightarrow \infty}y_{t}&=&\frac{a_{0}}{1-a_{1}}+\sum_{i=0}^{\infty}a_{1}^{i}\varepsilon_{t-i}
\end{eqnarray*}
\begin{eqnarray*}
Var(y_{t})&=&E\left[ (y_{t}-\mu)^{2}\right]=E\left[ (\varepsilon_{t}+a_{1}\varepsilon_{t-1}+a_{1}^{2}\varepsilon_{t-2}+\cdots)^{2}\right]\\
&=&\sigma^{2}\left[ (1+a_{1}^{2}+a_{1}^{4}+\cdots)\right]=\frac{\sigma^{2}}{1-a_{1}^{2}}
\end{eqnarray*}
\begin{eqnarray*}
Cov(y_{t}, y_{t-s})&=&E\left[ (y_{t}-\mu)(y_{t-s}-\mu)\right]\\
&=&E\left[ (\varepsilon_{t}+a_{1}\varepsilon_{t-1}+a_{1}^{2}\varepsilon_{t-2}+\cdots)(\varepsilon_{t-s}+a_{1}\varepsilon_{t-s-1}+a_{1}^{2}\varepsilon_{t-s-2}+\cdots)\right]\\
&=&E\left( a_{1}^{s}\varepsilon_{t-s}^{2}+a_{1}^{s+2}\varepsilon_{t-s-1}^{2}+a_{1}^{s+4}\varepsilon_{t-s-2}^{2}+\cdots\right)\\
&=&\sigma^{2}a_{1}^{s}(1+a_{1}^{2}+a_{1}^{4}+\cdots)\\
&=&\frac{\sigma^{2}a_{1}^{s}}{1-a_{1}^{2}}
\end{eqnarray*}
\end{itemize}

\subsubsection{The Autocorrelation Function of an AR(2) Process}
We assume that $a_{0}=0$, which implies that $E(y_{t})=0$. Adding or subtracting any constant from a variable does not change its variance, covariance, correlation coefficient, etc.

Using Yule-Walker equations: multiply the second-order D.E by $y_{t-s}$ for s$=0, 1, 2, \cdots$, and take expectations
\begin{eqnarray*}
Ey_{t}y_{t}&=&a_{1}Ey_{t-1}y_{t}+a_{2}Ey_{t-2}y_{t}+E\varepsilon_{t}y_{t}\\
Ey_{t}y_{t-1}&=&a_{1}Ey_{t-1}y_{t-1}+a_{2}Ey_{t-2}y_{t-1}+E\varepsilon_{t}y_{t-1}\\
Ey_{t}y_{t-2}&=&a_{1}Ey_{t-1}y_{t-2}+a_{2}Ey_{t-2}y_{t-2}+E\varepsilon_{t}y_{t-2}\\
&\vdots&\\
Ey_{t}y_{t-s}&=&a_{1}Ey_{t-1}y_{t-s}+a_{2}Ey_{t-2}y_{t-s}+E\varepsilon_{t}y_{t-s}\\
\end{eqnarray*}
By definition, the autocovariances of a stationary series are such 
\begin{eqnarray*}
Ey_{t}y_{t-s}&=&Ey_{t-s}y_{t}=Ey_{t-k}y_{t-k-s}=\gamma_{s}
\end{eqnarray*}
We also know that coefficient on $\varepsilon_{t}$ is unity so that $E\varepsilon_{t}y_{t}=\sigma^{2}$, and $E\varepsilon_{t}y_{t-s}=0$, so
\begin{eqnarray*}
\gamma_{0}&=&a_{1}\gamma_{1}+a_{2}\gamma_{2}+\sigma^{2}\\
\gamma_{1}&=&a_{1}\gamma_{0}+a_{2}\gamma_{1}\\
\gamma_{2}&=&a_{1}\gamma_{1}+a_{2}\gamma_{0}\\
&\vdots&\\
\gamma_{s}&=&a_{1}\gamma_{s-1}+a_{2}\gamma_{s-2}
\end{eqnarray*}
Now we can get the ACF
\begin{eqnarray*}
\rho_{1}&=&a_{1}\rho_{0}+a_{2}\rho_{1}\\
\rho_{s}&=&a_{1}\rho_{s-1}+a_{2}\rho_{s-2}
\end{eqnarray*}
We know $\rho_{0}=1$, so
\begin{eqnarray*}
\rho_{1}&=&a_{1}+a_{2}\rho_{1}\\
\rho_{1}&=&\frac{a_{1}}{1-a_{2}}
\end{eqnarray*}

\subsubsection{The Autocorrelation Function of an MA(1) Process}
Consider the MA(1) process $y_{t}=\varepsilon_{t}+\beta\varepsilon_{t-1}$

Applying the Yule-Walker equations 
\begin{eqnarray*}
\gamma_{0}&=&E(y_{t}y_{t})=E\left[ (\varepsilon_{t}+\beta\varepsilon_{t-1})(\varepsilon_{t}+\beta\varepsilon_{t-1})\right]=(1+\beta^{2})\sigma^{2}\\
\gamma_{1}&=&E(y_{t}y_{t-1})=E\left[ (\varepsilon_{t}+\beta\varepsilon_{t-1})(\varepsilon_{t-1}+\beta\varepsilon_{t-2})\right]=\beta^{2}\sigma^{2}\\
\gamma_{2}&=&E(y_{t}y_{t-2})=E\left[ (\varepsilon_{t}+\beta\varepsilon_{t-1})(\varepsilon_{t-2}+\beta\varepsilon_{t-3})\right]=0\\
\gamma_{s}&=&E(y_{t}y_{t-s})=E\left[ (\varepsilon_{t}+\beta\varepsilon_{t-1})(\varepsilon_{t-s}+\beta\varepsilon_{t-s-1})\right]=0 \ \ \forall t>2
\end{eqnarray*}

So the ACF of MA(1)
\begin{eqnarray*}
\rho_{0}&=&1\\
\rho_{1}&=&\frac{\gamma_{1}}{\gamma_{0}}=\frac{\beta^{2}}{1+\beta^{2}}\\
\rho_{s}&=&0 \ \ \forall t>1
\end{eqnarray*}

\subsubsection{The Autocorrelation Function of an ARMA(1,1) Process}
Consider the ARMA(1,1) $y_{t}=a_{1}y_{t-1}+\varepsilon_{t}+\beta_{1}\varepsilon_{t-1}$
\begin{eqnarray*}
Ey_{t}y_{t}&=&a_{1}Ey_{t-1}y_{t}+E\varepsilon_{t}y_{t}+\beta_{1}E\varepsilon_{t-1}y_{t} \Rightarrow \\
\gamma_{0}&=&a_{1}\gamma_{1}+\sigma^{2}+\beta_{1}(a_{1}+\beta_{1})\sigma^{2}\\
Ey_{t}y_{t-1}&=&a_{1}Ey_{t-1}y_{t-1}+E\varepsilon_{t}y_{t-1}+\beta_{1}E\varepsilon_{t-1}y_{t-1} \Rightarrow \\
\gamma_{1}&=&a_{1}\gamma_{0}+\beta_{1}\sigma^{2}\\
Ey_{t}y_{t-2}&=&a_{1}Ey_{t-1}y_{t-2}+E\varepsilon_{t}y_{t-2}+\beta_{1}E\varepsilon_{t-1}y_{t-2} \Rightarrow \\
\gamma_{2}&=&a_{1}\gamma_{1}\\
Ey_{t}y_{t-s}&=&a_{1}Ey_{t-1}y_{t-s}+E\varepsilon_{t}y_{t-s}+\beta_{1}E\varepsilon_{t-1}y_{t-s} \Rightarrow \\
\gamma_{s}&=&a_{1}\gamma_{s-1}
\end{eqnarray*}
Solve the equations and get 
\begin{eqnarray*}
\gamma_{0}&=&\frac{1+\beta_{1}^{2}+2a_{1}\beta_{1}}{1-a_{1}^{2}}\sigma^{2}\\
\gamma_{1}&=&\frac{(1+a_{1}\beta_{1})(a_{1}+\beta_{1})}{1-a_{1}^{2}}\sigma^{2}
\end{eqnarray*}
And the AFC 
\begin{eqnarray*}
\rho_{0}&=&1\\
\rho_{1}&=&\frac{\gamma_{1}}{\gamma_{0}}=\frac{(1+a_{1}\beta_{1})(a_{1}+\beta_{1})}{1+\beta_{1}^{2}+2a_{1}\beta_{1}}\\
\rho_{s}&=&a_{1}\rho_{s} \ \ \forall t>1
\end{eqnarray*}

\subsection{The Partial Autoorrelation Function}
In AR(1) process, $y_{t}$ and $y_{t-2}$ are correlated even though $y_{t-2}$ does not directly appear in the model. $\rho_{2}=corr(y_{t}, y_{t-1})\times corr(y_{t-1}, y_{t-2})=\rho_{1}^{2}$. All such "indirect" correlations are present in the ACF. In contrast, the partial autocorrelation between $y_{t}$ and $y_{t-s}$ climinates the effects of the intervening values $y_{t-1}$ through $y_{t-s+1}$.

Method to find the PACF:
\begin{enumerate}
\item Form the series $\{y_{t}^{\ast}\}$, where $y_{t}^{\ast}\equiv y_{t}-\mu$
\item Form the first-order autoregression equation:
	\begin{eqnarray*}
	y_{t}^{\ast}&=&\phi_{11}y_{t-1}^{\ast}+e_{i}\\
	y_{t}^{\ast}&=&\phi_{21}y_{t-1}^{\ast}+\phi_{22}y_{t-2}^{\ast}+e_{i}
	\end{eqnarray*}

where $\phi_{11}$ is the partial autocorrelation between $y_{t}$ and $y_{t-1}$, $\phi_{22}$ is the partial autocorrelation between $y_{t}$ and $y_{t-2}$. Repeating this process for all additional lags s yields the PACF.
\end{enumerate}

\begin{table}[!h]
\caption{Properties of the ACF and PACF}
\label{acf}
\begin{center}
\begin{tabular}{lll}\hline
Process		&		ACF		&PACF\\ \hline
White-noise& 	$\rho_{s}=0$&$\phi_{ss}=0$\\
AR(1): $a_{1}>0$&$\rho_{s}=a_{1}^{s}$ 	&$\phi_{11}=\rho_{1}$; $\phi_{ss}=0$ for $s\geq2$\\
AR(1): $a_{1}<0$&$\rho_{s}=a_{1}^{s}$ 	&$\phi_{11}=\rho_{1}$; $\phi_{ss}=0$ for $s\geq2$\\
AR(p)&Decays toward zero.  	&Spikes through lag p\\
	&Coefficients may oscillate&$\phi_{ss}=0$ for $s\geq p$\\
MA(1): $\beta>0$& $\rho_{s}=0, for \ s\geq 2$&Oscillating decay: $\phi_{11}>0$\\
MA(1): $\beta>0$& $\rho_{s}=0, for \ s\geq 2$&Decay: $\phi_{11}<0$ \\
ARMA(1, 1): $a_{1}>0$& 	&\\
ARMA(1, 1): $a_{1}<0$& 	&\\
ARMA(p, q)& 	&\\
\hline
\end{tabular}
\end{center}
\end{table}


\subsection{Sample Autoorrelations}
Given that a series is stationary, we can use the sample mean, variance and autocorrelations to estimate the parameters of the actual data-generating process.

The estimates of $\mu$, $\sigma^{2}$ and $\rho$:
\begin{eqnarray*}
\overline{y}&=&\frac{\sum_{t=1}^{T}y_{t}}{T}\\
\hat{\sigma^{2}}&=&\frac{\sum_{t=1}^{T}(y_{t}-\overline{y})^{2}}{T}\\
r_{s}&=&\frac{\sum_{t=s+1}^{T}(y_{t}-\overline{y})(y_{t-s}-\overline{y})}{\sum_{t=1}^{T}(y_{t}-\overline{y})^{2}}
\end{eqnarray*}

Box and Pierce used the sample autocorrelations to form the statistic
\begin{eqnarray*}
Q&=&T\sum_{k=1}^{s}r_{k}^{2}
\end{eqnarray*}

If the data are generated from a stationary ARMA process, Q is asymptotically $\chi^{2}(s)$ distribution. 

Ljung and Box test
\begin{eqnarray*}
Q&=&T(T+2)\sum_{k=1}^{s}\frac{r_{k}^{2}}{T-k} \sim \chi^{2}(s)
\end{eqnarray*}