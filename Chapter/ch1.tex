\section[Modelling Ecomomic Time Series]{Modelling Ecomomic Time Series}
\subsection{Data Generation Processes}
Let $\x_{t}$ be an $m\times 1$ vector of economic variables generated at time $t$. Such variables are typically inter0related both contemporaneously and across time. The collection $\{\x_{t},-\infty<t<\infty\}$ is called a vector-valued random sequence. 

The data generation process (DGP) is represented by the conditional density
\begin{eqnarray*}
D_{t}\left(\x_{t}\mid\mathcal{X}_{t-1}\right)
\end{eqnarray*}
where $\mathcal{X}_{t-1}=\sigma\left(\x_{t-1},\x_{t-2},\x_{t-3},\cdots\right)$. This is a shorthand for the $\sigma$-field representing knowledge of the past history of the system. Notice that
\begin{remark}
\item $D_{t}$ is allowed to depend on time, because the data are not assumed to be stationary.
\end{remark}

\subsection{DGPs and Models}
A dynamic econometric model is a family of functions of the data which are intended to mimic aspects of the DGP, either $D_{t}$ itself or functions derived from $D_{t}$ such as moments. Formally, a model is a family of functions
\begin{eqnarray*}
\{M\left(\x_{t},\x_{t-1},\x_{t-2},\cdots,\bm{d}_{t};\bm{\psi}\right),\bm{\psi}\in\Psi\},\Psi\subset\mathbb{R}^{p}
\end{eqnarray*}
In particular, let $M_{D}$ be a model of complete DGP.
\begin{itemize}
\item Parameters $\bm{\psi}$: $p$ in number; parameters are constants that are common to every $t$.
\item Parameter space: $\Psi$ denotes the set of admissible parameter values.
\item The vector $\bm{d}_{t}$ represents variables, treated as non-stochastic, which are intended to capture the changes in the DGP over time. 
\end{itemize}

The relationship between the DGP and the model is a difficult issue. The axiom of correct specification is the assumption that there exists a model element that is identical to the corresponding function of the DGP. $M_{D}$ is correctly specified if there exists $\bm{\psi}_{0}\in\Psi$ such that
\begin{eqnarray*}
M_{D}\left(\x_{t},\x_{t-1},\x_{t-2},\cdots,\bm{d}_{t};\bm{\psi}_{0}\right)&=&D_{t}\left(\x_{t}\mid\mathcal{X}_{t-1}\right)
\end{eqnarray*}
In general, correct specification in practical modelling exercises is an implausible assumption. 

\subsection{Sequence Properties}
\subsubsection{Stationarity}
A random sequence $\{\x_{t}\}$ is said to be \textit{stationary in the wide sense/covariance-stationary}
 if the mean, the variance and the sequence of $j$th-order autocovariances for $j>0$ are all independent of $t$. It is said to be \textit{stationary in the strict sense} if for every $k>0$, the joint distributions of all collections $\left(\x_{t},\x_{t+1},\x_{t+2},\cdots,\x_{t+k}\right)$ do not depend in any way on $t$.

\subsubsection{Mixing}
In a mixing sequence the realization of the sequence at time $t$ contains no information about the realization at either $t-j$ or $t+j$, when $j$ is sufficiently large. The mixing property ensures that points in the sequence appear randomly sampled when they are far enough apart.
\begin{remark}
Stationarity and mixing are quite distinct properties.
\end{remark}

\subsubsection{Ergodicity}
A stationary sequence having the property that a random event involving every member of the sequence always has probability either 0 or 1 is called \textit{ergodic}.
\begin{theorem}
If $\{x_{t}\}$ is a stationary ergodic sequence, and $E(x_1)$ exists, $\bar{x}_{n}\to E(x_{1})$ with probability 1.
\end{theorem}